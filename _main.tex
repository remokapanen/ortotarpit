% Options for packages loaded elsewhere
\PassOptionsToPackage{unicode}{hyperref}
\PassOptionsToPackage{hyphens}{url}
\documentclass[
]{book}
\usepackage{xcolor}
\usepackage{amsmath,amssymb}
\setcounter{secnumdepth}{5}
\usepackage{iftex}
\ifPDFTeX
  \usepackage[T1]{fontenc}
  \usepackage[utf8]{inputenc}
  \usepackage{textcomp} % provide euro and other symbols
\else % if luatex or xetex
  \usepackage{unicode-math} % this also loads fontspec
  \defaultfontfeatures{Scale=MatchLowercase}
  \defaultfontfeatures[\rmfamily]{Ligatures=TeX,Scale=1}
\fi
\usepackage{lmodern}
\ifPDFTeX\else
  % xetex/luatex font selection
\fi
% Use upquote if available, for straight quotes in verbatim environments
\IfFileExists{upquote.sty}{\usepackage{upquote}}{}
\IfFileExists{microtype.sty}{% use microtype if available
  \usepackage[]{microtype}
  \UseMicrotypeSet[protrusion]{basicmath} % disable protrusion for tt fonts
}{}
\makeatletter
\@ifundefined{KOMAClassName}{% if non-KOMA class
  \IfFileExists{parskip.sty}{%
    \usepackage{parskip}
  }{% else
    \setlength{\parindent}{0pt}
    \setlength{\parskip}{6pt plus 2pt minus 1pt}}
}{% if KOMA class
  \KOMAoptions{parskip=half}}
\makeatother
\usepackage{color}
\usepackage{fancyvrb}
\newcommand{\VerbBar}{|}
\newcommand{\VERB}{\Verb[commandchars=\\\{\}]}
\DefineVerbatimEnvironment{Highlighting}{Verbatim}{commandchars=\\\{\}}
% Add ',fontsize=\small' for more characters per line
\usepackage{framed}
\definecolor{shadecolor}{RGB}{248,248,248}
\newenvironment{Shaded}{\begin{snugshade}}{\end{snugshade}}
\newcommand{\AlertTok}[1]{\textcolor[rgb]{0.94,0.16,0.16}{#1}}
\newcommand{\AnnotationTok}[1]{\textcolor[rgb]{0.56,0.35,0.01}{\textbf{\textit{#1}}}}
\newcommand{\AttributeTok}[1]{\textcolor[rgb]{0.13,0.29,0.53}{#1}}
\newcommand{\BaseNTok}[1]{\textcolor[rgb]{0.00,0.00,0.81}{#1}}
\newcommand{\BuiltInTok}[1]{#1}
\newcommand{\CharTok}[1]{\textcolor[rgb]{0.31,0.60,0.02}{#1}}
\newcommand{\CommentTok}[1]{\textcolor[rgb]{0.56,0.35,0.01}{\textit{#1}}}
\newcommand{\CommentVarTok}[1]{\textcolor[rgb]{0.56,0.35,0.01}{\textbf{\textit{#1}}}}
\newcommand{\ConstantTok}[1]{\textcolor[rgb]{0.56,0.35,0.01}{#1}}
\newcommand{\ControlFlowTok}[1]{\textcolor[rgb]{0.13,0.29,0.53}{\textbf{#1}}}
\newcommand{\DataTypeTok}[1]{\textcolor[rgb]{0.13,0.29,0.53}{#1}}
\newcommand{\DecValTok}[1]{\textcolor[rgb]{0.00,0.00,0.81}{#1}}
\newcommand{\DocumentationTok}[1]{\textcolor[rgb]{0.56,0.35,0.01}{\textbf{\textit{#1}}}}
\newcommand{\ErrorTok}[1]{\textcolor[rgb]{0.64,0.00,0.00}{\textbf{#1}}}
\newcommand{\ExtensionTok}[1]{#1}
\newcommand{\FloatTok}[1]{\textcolor[rgb]{0.00,0.00,0.81}{#1}}
\newcommand{\FunctionTok}[1]{\textcolor[rgb]{0.13,0.29,0.53}{\textbf{#1}}}
\newcommand{\ImportTok}[1]{#1}
\newcommand{\InformationTok}[1]{\textcolor[rgb]{0.56,0.35,0.01}{\textbf{\textit{#1}}}}
\newcommand{\KeywordTok}[1]{\textcolor[rgb]{0.13,0.29,0.53}{\textbf{#1}}}
\newcommand{\NormalTok}[1]{#1}
\newcommand{\OperatorTok}[1]{\textcolor[rgb]{0.81,0.36,0.00}{\textbf{#1}}}
\newcommand{\OtherTok}[1]{\textcolor[rgb]{0.56,0.35,0.01}{#1}}
\newcommand{\PreprocessorTok}[1]{\textcolor[rgb]{0.56,0.35,0.01}{\textit{#1}}}
\newcommand{\RegionMarkerTok}[1]{#1}
\newcommand{\SpecialCharTok}[1]{\textcolor[rgb]{0.81,0.36,0.00}{\textbf{#1}}}
\newcommand{\SpecialStringTok}[1]{\textcolor[rgb]{0.31,0.60,0.02}{#1}}
\newcommand{\StringTok}[1]{\textcolor[rgb]{0.31,0.60,0.02}{#1}}
\newcommand{\VariableTok}[1]{\textcolor[rgb]{0.00,0.00,0.00}{#1}}
\newcommand{\VerbatimStringTok}[1]{\textcolor[rgb]{0.31,0.60,0.02}{#1}}
\newcommand{\WarningTok}[1]{\textcolor[rgb]{0.56,0.35,0.01}{\textbf{\textit{#1}}}}
\usepackage{longtable,booktabs,array}
\usepackage{calc} % for calculating minipage widths
% Correct order of tables after \paragraph or \subparagraph
\usepackage{etoolbox}
\makeatletter
\patchcmd\longtable{\par}{\if@noskipsec\mbox{}\fi\par}{}{}
\makeatother
% Allow footnotes in longtable head/foot
\IfFileExists{footnotehyper.sty}{\usepackage{footnotehyper}}{\usepackage{footnote}}
\makesavenoteenv{longtable}
\usepackage{graphicx}
\makeatletter
\newsavebox\pandoc@box
\newcommand*\pandocbounded[1]{% scales image to fit in text height/width
  \sbox\pandoc@box{#1}%
  \Gscale@div\@tempa{\textheight}{\dimexpr\ht\pandoc@box+\dp\pandoc@box\relax}%
  \Gscale@div\@tempb{\linewidth}{\wd\pandoc@box}%
  \ifdim\@tempb\p@<\@tempa\p@\let\@tempa\@tempb\fi% select the smaller of both
  \ifdim\@tempa\p@<\p@\scalebox{\@tempa}{\usebox\pandoc@box}%
  \else\usebox{\pandoc@box}%
  \fi%
}
% Set default figure placement to htbp
\def\fps@figure{htbp}
\makeatother
\setlength{\emergencystretch}{3em} % prevent overfull lines
\providecommand{\tightlist}{%
  \setlength{\itemsep}{0pt}\setlength{\parskip}{0pt}}
\usepackage[]{natbib}
\bibliographystyle{plainnat}
\usepackage{booktabs}
\usepackage{bookmark}
\IfFileExists{xurl.sty}{\usepackage{xurl}}{} % add URL line breaks if available
\urlstyle{same}
\hypersetup{
  pdftitle={Ortopedia, traumatologia \& käsikirurgia -tärpit},
  pdfauthor={Remo Kapanen},
  hidelinks,
  pdfcreator={LaTeX via pandoc}}

\title{Ortopedia, traumatologia \& käsikirurgia -tärpit}
\author{Remo Kapanen}
\date{2025-11-10}

\usepackage{amsthm}
\newtheorem{theorem}{Theorem}[chapter]
\newtheorem{lemma}{Lemma}[chapter]
\newtheorem{corollary}{Corollary}[chapter]
\newtheorem{proposition}{Proposition}[chapter]
\newtheorem{conjecture}{Conjecture}[chapter]
\theoremstyle{definition}
\newtheorem{definition}{Definition}[chapter]
\theoremstyle{definition}
\newtheorem{example}{Example}[chapter]
\theoremstyle{definition}
\newtheorem{exercise}{Exercise}[chapter]
\theoremstyle{definition}
\newtheorem{hypothesis}{Hypothesis}[chapter]
\theoremstyle{remark}
\newtheorem*{remark}{Remark}
\newtheorem*{solution}{Solution}
\begin{document}
\maketitle

{
\setcounter{tocdepth}{1}
\tableofcontents
}
\chapter{Wiki}\label{wiki}

Tärpit ovat wikissä formatoitu aika rumasti ja niiden navigointi on vaikeaa, koska kysymyksiä ei ole numeroitu oikein vastausten mukaan ja uudemmissa ei edes ole vastauksia. Vanhemmissa tärpeissä on myös runsaasti virheellisiä vastauksia.

Toivottavasti tämä rebuildi parantaa tärppien läpikäymisprosessiasi.

\chapter{2020 (Fortis )}\label{fortis}

\section{Solisluun murtumat}\label{solisluun-murtumat}

Valitse yksi:

\begin{itemize}
\tightlist
\item
  \begin{enumerate}
  \def\labelenumi{\alph{enumi}.}
  \tightlist
  \item
    Luutumattomuusriski on 8-26\%, mutta osa näistä on oireettomia
  \end{enumerate}
\item
  \begin{enumerate}
  \def\labelenumi{\alph{enumi}.}
  \setcounter{enumi}{1}
  \tightlist
  \item
    Keskikolmanneksen murtumat ovat yleisimpiä
  \end{enumerate}
\item
  \begin{enumerate}
  \def\labelenumi{\alph{enumi}.}
  \setcounter{enumi}{2}
  \tightlist
  \item
    Kaikki vastausvaihtoehdot ovat oikein
  \end{enumerate}
\item
  \begin{enumerate}
  \def\labelenumi{\alph{enumi}.}
  \setcounter{enumi}{3}
  \tightlist
  \item
    Keskikolmanneksen murtumat hoidetaan yleensä konservatiivisesti
  \end{enumerate}
\end{itemize}

\begin{solution}
\leavevmode

Vastaus

\begin{verbatim}
c
\end{verbatim}

\end{solution}

\section{Päivystyksessä (Kuusamon terveyskeskus) lähtisin hoitamaan kyseistä murtumaa ensin}\label{puxe4ivystyksessuxe4-kuusamon-terveyskeskus-luxe4htisin-hoitamaan-kyseistuxe4-murtumaa-ensin}

\pandocbounded{\includegraphics[keepaspectratio]{images/K2_2020.png}}

Valitse yksi:

\begin{itemize}
\tightlist
\item
  \begin{enumerate}
  \def\labelenumi{\alph{enumi}.}
  \tightlist
  \item
    Repositiolla paikallis-puudutuksessa ja dorsaalisella kipsilastalla
  \end{enumerate}
\item
  \begin{enumerate}
  \def\labelenumi{\alph{enumi}.}
  \setcounter{enumi}{1}
  \tightlist
  \item
    Kirjoittaisin lähetteen erikoissairaanhoitoon (OYS)
  \end{enumerate}
\item
  \begin{enumerate}
  \def\labelenumi{\alph{enumi}.}
  \setcounter{enumi}{2}
  \tightlist
  \item
    Repositiolla ja kulmakipsillä
  \end{enumerate}
\item
  \begin{enumerate}
  \def\labelenumi{\alph{enumi}.}
  \setcounter{enumi}{3}
  \tightlist
  \item
    Irroitettavalla rannelastalla
  \end{enumerate}
\end{itemize}

\begin{solution}
\leavevmode

Vastaus

\begin{verbatim}
a
\end{verbatim}

\end{solution}

\section{Minkälainen nilkkavamma?}\label{minkuxe4lainen-nilkkavamma}

\pandocbounded{\includegraphics[keepaspectratio]{images/K3_2020.png}}

Valitse yksi:

\begin{itemize}
\tightlist
\item
  \begin{enumerate}
  \def\labelenumi{\alph{enumi}.}
  \tightlist
  \item
    Weber B-tyypin lateraalimalleolin murtuma, ehyt syndesmoosin etulehti ja deltaligamentin repeämä
  \end{enumerate}
\item
  \begin{enumerate}
  \def\labelenumi{\alph{enumi}.}
  \setcounter{enumi}{1}
  \tightlist
  \item
    Weber B-tyypin lateraalimalleolin murtuma ja FTA-ligamentin repeämä
  \end{enumerate}
\item
  \begin{enumerate}
  \def\labelenumi{\alph{enumi}.}
  \setcounter{enumi}{2}
  \tightlist
  \item
    Weber A-tyypin lateraalimalleolin murtuma ja FTA-ligamentin repeämä
  \end{enumerate}
\item
  \begin{enumerate}
  \def\labelenumi{\alph{enumi}.}
  \setcounter{enumi}{3}
  \tightlist
  \item
    Weber A-tyypin lateraalimalleolin murtuma ja deltaligamentin repeämä
  \end{enumerate}
\end{itemize}

\begin{solution}
\leavevmode

Vastaus

\begin{verbatim}
a
\end{verbatim}

\begin{itemize}
\item
  Weber B = murtuma nivelraon tasolla (monissa lähteissä sanotaan syndesmoositasolla)
\end{itemize}

\section{Mikä seuraavista rakenteista on FTA?}\label{mikuxe4-seuraavista-rakenteista-on-fta}

\pandocbounded{\includegraphics[keepaspectratio]{images/K4_2020.png}}

Valitse yksi:

\begin{itemize}
\tightlist
\item
  \begin{enumerate}
  \def\labelenumi{\alph{enumi}.}
  \tightlist
  \item
    2
  \end{enumerate}
\item
  \begin{enumerate}
  \def\labelenumi{\alph{enumi}.}
  \setcounter{enumi}{1}
  \tightlist
  \item
    4
  \end{enumerate}
\item
  \begin{enumerate}
  \def\labelenumi{\alph{enumi}.}
  \setcounter{enumi}{2}
  \tightlist
  \item
    1
  \end{enumerate}
\item
  \begin{enumerate}
  \def\labelenumi{\alph{enumi}.}
  \setcounter{enumi}{3}
  \tightlist
  \item
    3
  \end{enumerate}
\end{itemize}

\begin{solution}
\leavevmode

Vastaus

\begin{verbatim}
a
\end{verbatim}

\end{solution}

\section{Kuvassa näkyvä jalkaterän liike tulee pääasiallisesti}\label{kuvassa-nuxe4kyvuxe4-jalkateruxe4n-liike-tulee-puxe4uxe4asiallisesti}

\pandocbounded{\includegraphics[keepaspectratio]{images/K5_2020.png}}

Valitse yksi:

\begin{itemize}
\tightlist
\item
  \begin{enumerate}
  \def\labelenumi{\alph{enumi}.}
  \tightlist
  \item
    TC-nivelestä
  \end{enumerate}
\item
  \begin{enumerate}
  \def\labelenumi{\alph{enumi}.}
  \setcounter{enumi}{1}
  \tightlist
  \item
    Lisfrancin nivelestä
  \end{enumerate}
\item
  \begin{enumerate}
  \def\labelenumi{\alph{enumi}.}
  \setcounter{enumi}{2}
  \tightlist
  \item
    TMT nivelistä
  \end{enumerate}
\item
  \begin{enumerate}
  \def\labelenumi{\alph{enumi}.}
  \setcounter{enumi}{3}
  \tightlist
  \item
    Subtalaarinivelestä
  \end{enumerate}
\end{itemize}

\begin{solution}
\leavevmode

Vastaus

\begin{verbatim}
d
\end{verbatim}

\end{solution}

\section{Mikä seuraavista rakenteista on tibialis posterior jänne?}\label{mikuxe4-seuraavista-rakenteista-on-tibialis-posterior-juxe4nne}

\pandocbounded{\includegraphics[keepaspectratio]{images/K6_2020.png}}

Valitse yksi:

\begin{itemize}
\tightlist
\item
  \begin{enumerate}
  \def\labelenumi{\alph{enumi}.}
  \tightlist
  \item
    2
  \end{enumerate}
\item
  \begin{enumerate}
  \def\labelenumi{\alph{enumi}.}
  \setcounter{enumi}{1}
  \tightlist
  \item
    3
  \end{enumerate}
\item
  \begin{enumerate}
  \def\labelenumi{\alph{enumi}.}
  \setcounter{enumi}{2}
  \tightlist
  \item
    1
  \end{enumerate}
\item
  \begin{enumerate}
  \def\labelenumi{\alph{enumi}.}
  \setcounter{enumi}{3}
  \tightlist
  \item
    4
  \end{enumerate}
\end{itemize}

\begin{solution}
\leavevmode

Vastaus

\begin{verbatim}
d
\end{verbatim}

\end{solution}

\section{Hamstring repeämät}\label{hamstring-repeuxe4muxe4t}

Valitse yksi:

\begin{itemize}
\tightlist
\item
  \begin{enumerate}
  \def\labelenumi{\alph{enumi}.}
  \tightlist
  \item
    Leikkaushoidon tulokset ovat epävarmat, jos kyseessä on jänteiden repeytyminen irti insertiosta (Gradus III vammat)
  \end{enumerate}
\item
  \begin{enumerate}
  \def\labelenumi{\alph{enumi}.}
  \setcounter{enumi}{1}
  \tightlist
  \item
    MRI on aina tarpeellinen diagnostiikassa
  \end{enumerate}
\item
  \begin{enumerate}
  \def\labelenumi{\alph{enumi}.}
  \setcounter{enumi}{2}
  \tightlist
  \item
    Tavataan käytännössä vain urheilijoilla
  \end{enumerate}
\item
  \begin{enumerate}
  \def\labelenumi{\alph{enumi}.}
  \setcounter{enumi}{3}
  \tightlist
  \item
    Täydellinen repeämä (Gradus III vamma) voi syntyä keski-iässä liukastumistapaturman yhteydessä
  \end{enumerate}
\item
  \begin{enumerate}
  \def\labelenumi{\alph{enumi}.}
  \setcounter{enumi}{4}
  \tightlist
  \item
    Etureiden hematooma on usein avuksi diagnostiikassa
  \end{enumerate}
\end{itemize}

\begin{solution}
\leavevmode

Vastaus

\begin{verbatim}
d
\end{verbatim}

\end{solution}

\section{Mitä kuvassa testataan (potilas makuulla vatsallaan)?}\label{mituxe4-kuvassa-testataan-potilas-makuulla-vatsallaan}

\pandocbounded{\includegraphics[keepaspectratio]{images/K8_2020.png}}

Valitse yksi:

\begin{itemize}
\tightlist
\item
  \begin{enumerate}
  \def\labelenumi{\alph{enumi}.}
  \tightlist
  \item
    Aitiopaine oireyhtymää
  \end{enumerate}
\item
  \begin{enumerate}
  \def\labelenumi{\alph{enumi}.}
  \setcounter{enumi}{1}
  \tightlist
  \item
    Akilliesjänne repeämää
  \end{enumerate}
\item
  \begin{enumerate}
  \def\labelenumi{\alph{enumi}.}
  \setcounter{enumi}{2}
  \tightlist
  \item
    Suralis hermon toimintaa
  \end{enumerate}
\item
  \begin{enumerate}
  \def\labelenumi{\alph{enumi}.}
  \setcounter{enumi}{3}
  \tightlist
  \item
    Hamstring repeämää
  \end{enumerate}
\end{itemize}

\begin{solution}
\leavevmode

Vastaus

\begin{verbatim}
b
\end{verbatim}

\end{solution}

\section{Eturistisiteen repeämä}\label{eturistisiteen-repeuxe4muxe4}

Valitse yksi:

\begin{itemize}
\tightlist
\item
  \begin{enumerate}
  \def\labelenumi{\alph{enumi}.}
  \tightlist
  \item
    Vammaan liittyy usein MRI-kuvissa näkyvä kantavan nivelpinnan rustovaurio ja luukontuusio
  \end{enumerate}
\item
  \begin{enumerate}
  \def\labelenumi{\alph{enumi}.}
  \setcounter{enumi}{1}
  \tightlist
  \item
    On useimmiten yksittäinen vamma eikä kuulu ns. ''unhappy trias'' vammaan
  \end{enumerate}
\item
  \begin{enumerate}
  \def\labelenumi{\alph{enumi}.}
  \setcounter{enumi}{2}
  \tightlist
  \item
    ACL on kestävä ligamentti, joten sen repeämä vaatii aina kontaktin toisen pelaajan kanssa joukkuepelilajeissa
  \end{enumerate}
\item
  \begin{enumerate}
  \def\labelenumi{\alph{enumi}.}
  \setcounter{enumi}{3}
  \tightlist
  \item
    Vamman jälkeisenä päivänä punktiossa saadaan tyypillisesti runsaasti nivelnestettä
  \end{enumerate}
\end{itemize}

\begin{solution}
\leavevmode

Vastaus

\begin{verbatim}
a
\end{verbatim}

\begin{itemize}
\item
  b: Usein vammautuu toisten nivelsiteiden tai nivelkierukan kanssa; unhappy triad = ACL-vamma+MCL-vamma+kierukkavamma\\
\item
  c: Syntyy tyypillisesti valgisoivan kiertovääntövamman seurauksena; ei siis tarvitse suoraa kontaktia
\item
  d: Vammassa saadaan tyypillisesti verta punktoitua; ACL-repeämä on yleisin veripolven aiheuttaja aikuisilla
\end{itemize}

\section{Supraspinatuksen toimintaa voi testata}\label{supraspinatuksen-toimintaa-voi-testata}

Valitse yksi:

\begin{itemize}
\tightlist
\item
  \begin{enumerate}
  \def\labelenumi{\alph{enumi}.}
  \tightlist
  \item
    Pitotestillä kyynärpää 90° kulmassa vaakatasossa
  \end{enumerate}
\item
  \begin{enumerate}
  \def\labelenumi{\alph{enumi}.}
  \setcounter{enumi}{1}
  \tightlist
  \item
    Vastustetulla abduktiolla pikkurilli ylöspäin hiukan vaakatason alapuolella
  \end{enumerate}
\item
  \begin{enumerate}
  \def\labelenumi{\alph{enumi}.}
  \setcounter{enumi}{2}
  \tightlist
  \item
    Vastustetulla etuelevaatiolla
  \end{enumerate}
\item
  \begin{enumerate}
  \def\labelenumi{\alph{enumi}.}
  \setcounter{enumi}{3}
  \tightlist
  \item
    Pyytämällä potilasta viemään käsi selän taakse ja irrottamaan käsi ihosta
  \end{enumerate}
\end{itemize}

\begin{solution}
\leavevmode

Vastaus

\begin{verbatim}
b
\end{verbatim}

\end{solution}

\section{Mikä seuraavista rakenteista on peroneus longus jänne?}\label{mikuxe4-seuraavista-rakenteista-on-peroneus-longus-juxe4nne}

\pandocbounded{\includegraphics[keepaspectratio]{images/K11_2020.png}}

Valitse yksi:

\begin{itemize}
\tightlist
\item
  \begin{enumerate}
  \def\labelenumi{\alph{enumi}.}
  \tightlist
  \item
    4
  \end{enumerate}
\item
  \begin{enumerate}
  \def\labelenumi{\alph{enumi}.}
  \setcounter{enumi}{1}
  \tightlist
  \item
    3
  \end{enumerate}
\item
  \begin{enumerate}
  \def\labelenumi{\alph{enumi}.}
  \setcounter{enumi}{2}
  \tightlist
  \item
    2
  \end{enumerate}
\item
  \begin{enumerate}
  \def\labelenumi{\alph{enumi}.}
  \setcounter{enumi}{3}
  \tightlist
  \item
    1
  \end{enumerate}
\end{itemize}

\begin{solution}
\leavevmode

Vastaus

\begin{verbatim}
b
\end{verbatim}

\end{solution}

\section{Mistä murtumasta on todennäköisimmin kyse?}\label{mistuxe4-murtumasta-on-todennuxe4kuxf6isimmin-kyse}

\pandocbounded{\includegraphics[keepaspectratio]{images/K12_2020.png}}

Valitse yksi:

\begin{itemize}
\tightlist
\item
  \begin{enumerate}
  \def\labelenumi{\alph{enumi}.}
  \tightlist
  \item
    Neer yhden fragmentin murtumasta
  \end{enumerate}
\item
  \begin{enumerate}
  \def\labelenumi{\alph{enumi}.}
  \setcounter{enumi}{1}
  \tightlist
  \item
    Kaikki muut vastausvaihtoehdot ovat väärin
  \end{enumerate}
\item
  \begin{enumerate}
  \def\labelenumi{\alph{enumi}.}
  \setcounter{enumi}{2}
  \tightlist
  \item
    Isoloidusta tuberculum majus murtumasta
  \end{enumerate}
\item
  \begin{enumerate}
  \def\labelenumi{\alph{enumi}.}
  \setcounter{enumi}{3}
  \tightlist
  \item
    Neer kolmen fragmentin murtumasta
  \end{enumerate}
\end{itemize}

\begin{solution}
\leavevmode

Vastaus

\begin{verbatim}
d 
 
\end{verbatim}

\begin{itemize}
\item
  Tuberculum majus on dislokoitunut \textgreater5mm (usein käytetään \textgreater5mm Neerin luokituksessa dislokoitumisrajana tuberculum majuksen suhteen; muut 1cm ja/tai \textgreater45 astetta angulaatiota). On myös murtumalinja kirurgisessa kaulassa ja se on dislokoitunut suhteessa varteen. On siis todettavissa kolme erillistä fragmenttia (tuberculum majus, caput ja varsi).
\end{itemize}

\section{Mikä murtuma?}\label{mikuxe4-murtuma}

\pandocbounded{\includegraphics[keepaspectratio]{images/K13_2020.png}}
Valitse yksi:

\begin{itemize}
\tightlist
\item
  \begin{enumerate}
  \def\labelenumi{\alph{enumi}.}
  \tightlist
  \item
    Dorsaalinen Barton
  \end{enumerate}
\item
  \begin{enumerate}
  \def\labelenumi{\alph{enumi}.}
  \setcounter{enumi}{1}
  \tightlist
  \item
    Smith
  \end{enumerate}
\item
  \begin{enumerate}
  \def\labelenumi{\alph{enumi}.}
  \setcounter{enumi}{2}
  \tightlist
  \item
    Volaarinen Barton
  \end{enumerate}
\item
  \begin{enumerate}
  \def\labelenumi{\alph{enumi}.}
  \setcounter{enumi}{3}
  \tightlist
  \item
    Die punch
  \end{enumerate}
\end{itemize}

\begin{solution}
\leavevmode

Vastaus

\begin{verbatim}
d
\end{verbatim}

\end{solution}

\section{Tyypillisen radiusmurtuman käypähoito suositus}\label{tyypillisen-radiusmurtuman-kuxe4ypuxe4hoito-suositus}

Valitse yksi:

\begin{itemize}
\tightlist
\item
  \begin{enumerate}
  \def\labelenumi{\alph{enumi}.}
  \tightlist
  \item
    Värttinäluun alaosan murtumasta toipuminen kestää 3-4 kk, jolloin voidaan arvioida lopputulos
  \end{enumerate}
\item
  \begin{enumerate}
  \def\labelenumi{\alph{enumi}.}
  \setcounter{enumi}{1}
  \tightlist
  \item
    Leikkaushoitoa suositellaan yli 65-vuotiaille, jos asento huononee kipsihoidon yhteydessä hyväksyttyjen raja-arvojen ulkopuolelle
  \end{enumerate}
\item
  \begin{enumerate}
  \def\labelenumi{\alph{enumi}.}
  \setcounter{enumi}{2}
  \tightlist
  \item
    Murtumien ensisijainen hoitopaikka on erikoissairaanhoito
  \end{enumerate}
\item
  \begin{enumerate}
  \def\labelenumi{\alph{enumi}.}
  \setcounter{enumi}{3}
  \tightlist
  \item
    Kipsihoito voidaan toteuttaa funktioasentoisella kipsilasta (ei volaarifleksiota, pronaatiota eikä ulnaarideviaatiota)
  \end{enumerate}
\end{itemize}

\begin{solution}
\leavevmode

Vastaus

\begin{verbatim}
d
\end{verbatim}

\begin{itemize}
\item
  Wikissä vastaukseksi on laitettu b, mutta näin kirjoitetaan hoito-ohjeissa: ``Yli 65-vuotiaalla radiologinen löydös ei ennusta toiminnallista lopputulosta eikä kirurgisesta hoidosta todennäköisesti ole hyötyä, vaikkei murtuman asento olisi hyväksyttävä tai se heikkenisi kipsihoidon aikana hyväksyttävien raja-arvojen ulkopuolelle. Kirurgista hoitoa voidaan harkita myös aktiivisilla ikääntyvillä, joilla on suuret vaatimukset yläraajan toimintakyvyn suhteen. Sen tarve tulee harkita näillä potilailla huolellisesti ja tapauskohtaisesti''
\item
  Ranteen tyyppimurtuman kipsihoito juuri on d-kohdassa kuvailtu funktioasentoinen kipsilasta, jossa 0-20 asteen ekstensio, eikä muuta vääntelyä käden asennon suhteen.
\end{itemize}

\section{Kuvan nilkassa}\label{kuvan-nilkassa}

\pandocbounded{\includegraphics[keepaspectratio]{imgaes/K15_2020.png}}

Valitse yksi:

\begin{itemize}
\tightlist
\item
  \begin{enumerate}
  \def\labelenumi{\alph{enumi}.}
  \tightlist
  \item
    Talus siirtynyt lateraalisesti
  \end{enumerate}
\item
  \begin{enumerate}
  \def\labelenumi{\alph{enumi}.}
  \setcounter{enumi}{1}
  \tightlist
  \item
    Kaikki vastausvaihtoehdot ovat oikein
  \end{enumerate}
\item
  \begin{enumerate}
  \def\labelenumi{\alph{enumi}.}
  \setcounter{enumi}{2}
  \tightlist
  \item
    Mediaalinen nivelrako on levinnyt
  \end{enumerate}
\item
  \begin{enumerate}
  \def\labelenumi{\alph{enumi}.}
  \setcounter{enumi}{3}
  \tightlist
  \item
    Lateraalimalleolissa murtuma
  \end{enumerate}
\end{itemize}

\begin{solution}
\leavevmode

Vastaus

\begin{verbatim}
b
\end{verbatim}

\end{solution}

\section{Lonkkamurtuma}\label{lonkkamurtuma}

Valitse yksi:

\begin{itemize}
\tightlist
\item
  \begin{enumerate}
  \def\labelenumi{\alph{enumi}.}
  \tightlist
  \item
    Avaskulaarisen nekroosin riski on suurempi pertrokanteerisissa murtumissa kuin reisiluun kaulan murtumissa
  \end{enumerate}
\item
  \begin{enumerate}
  \def\labelenumi{\alph{enumi}.}
  \setcounter{enumi}{1}
  \tightlist
  \item
    Yleisin murtumatyyppi on subtrokanteerinen murtuma
  \end{enumerate}
\item
  \begin{enumerate}
  \def\labelenumi{\alph{enumi}.}
  \setcounter{enumi}{2}
  \tightlist
  \item
    On tärkeää leikata mahdollisimman pian (1-2 vrk sisällä vammasta), koska sillä on vaikutus potilaan ennusteeseen
  \end{enumerate}
\item
  \begin{enumerate}
  \def\labelenumi{\alph{enumi}.}
  \setcounter{enumi}{3}
  \tightlist
  \item
    Riski saada lonkkamurtuma on hoitokodissa asuvilla pienempi kuin muulla samanikäisellä väestöllä
  \end{enumerate}
\end{itemize}

\begin{solution}
\leavevmode

Vastaus

\begin{verbatim}
c
\end{verbatim}

\end{solution}

\section{Olkanivelen sijoiltaanmeno}\label{olkanivelen-sijoiltaanmeno}

Valitse yksi:

\begin{itemize}
\tightlist
\item
  \begin{enumerate}
  \def\labelenumi{\alph{enumi}.}
  \tightlist
  \item
    Reposition jälkeen voimakasta ulkokiertoa on syytä välttää 6 viikon ajan
  \end{enumerate}
\item
  \begin{enumerate}
  \def\labelenumi{\alph{enumi}.}
  \setcounter{enumi}{1}
  \tightlist
  \item
    Liitännäisvammana on lähes aina leikkaushoitoa vaativa kiertäjäkalvosimen repeämä
  \end{enumerate}
\item
  \begin{enumerate}
  \def\labelenumi{\alph{enumi}.}
  \setcounter{enumi}{2}
  \tightlist
  \item
    Caput humeri luksoituu yleensä taaksepäin
  \end{enumerate}
\item
  \begin{enumerate}
  \def\labelenumi{\alph{enumi}.}
  \setcounter{enumi}{3}
  \tightlist
  \item
    Sijoiltaanmenon yhteydessä yleisimmin vaurioituva hermo on nervus radialis
  \end{enumerate}
\end{itemize}

\begin{solution}
\leavevmode

Vastaus

\begin{verbatim}
a (tosin ei enää) 
\end{verbatim}

\begin{itemize}
\item
  Vielä 2020 (kysymyksen tekovuosi) oli tapana vältellä ulkokiertoa 6 viikon ajan olkanivelen luksaation reposition jälkeen. Nykyään näin pitkää rajoitusta ei ilmeisesti enää ole. Ohje kuitenkin löytyy vielä Terveysportista olkanivelen sijoiltaanmeno -artikkelista.
\end{itemize}

\section{Subtrokanteerinen lonkkamurtuma}\label{subtrokanteerinen-lonkkamurtuma}

Valitse yksi:

\begin{itemize}
\tightlist
\item
  \begin{enumerate}
  \def\labelenumi{\alph{enumi}.}
  \tightlist
  \item
    Johtaa usein päävaltimoiden (a.femoralis profunda) vaurioitumiseen
  \end{enumerate}
\item
  \begin{enumerate}
  \def\labelenumi{\alph{enumi}.}
  \setcounter{enumi}{1}
  \tightlist
  \item
    Tämän murtuman ydinnaulaus johtaa usein rasvaemboliaan
  \end{enumerate}
\item
  \begin{enumerate}
  \def\labelenumi{\alph{enumi}.}
  \setcounter{enumi}{2}
  \tightlist
  \item
    Korkeaenerginen, tyypillisesti pirstaleinen murtuma, joka vaatii yleensä avointa reduktiota ja levytystä
  \end{enumerate}
\item
  \begin{enumerate}
  \def\labelenumi{\alph{enumi}.}
  \setcounter{enumi}{3}
  \tightlist
  \item
    Vanhuksilla kliiniseen kuvaan kuuluu raajan lyhentyminen, ulkorotaatio, paikallinen turvotus ja hypovolemiariski
  \end{enumerate}
\end{itemize}

\begin{solution}
\leavevmode

Vastaus

\begin{verbatim}
d
\end{verbatim}

\begin{itemize}
\item
  Kuten muihinkin lonkkamurtumiin
\end{itemize}

\section{Luunmurtuman paraneminen}\label{luunmurtuman-paraneminen}

Valitse yksi:

\begin{itemize}
\tightlist
\item
  \begin{enumerate}
  \def\labelenumi{\alph{enumi}.}
  \tightlist
  \item
    Pehmytosavauriot hidastavat luun paranemista
  \end{enumerate}
\item
  \begin{enumerate}
  \def\labelenumi{\alph{enumi}.}
  \setcounter{enumi}{1}
  \tightlist
  \item
    Putkiluu paranee nopeammin kuin hohkaluu
  \end{enumerate}
\item
  \begin{enumerate}
  \def\labelenumi{\alph{enumi}.}
  \setcounter{enumi}{2}
  \tightlist
  \item
    Murtumapäiden kompressio (kompressiolevytys) nopeuttaa luun paranemista
  \end{enumerate}
\item
  \begin{enumerate}
  \def\labelenumi{\alph{enumi}.}
  \setcounter{enumi}{3}
  \tightlist
  \item
    Dislokoitumattomat nivelen sisäiset murtumat pyritään ensisijaisesti leikkaamaan hyvän lopputuloksen turvaamiseksi
  \end{enumerate}
\end{itemize}

\begin{solution}
\leavevmode

Vastaus

\begin{verbatim}
a
\end{verbatim}

\end{solution}

\section{Atyyppinen reisiluun murtuma}\label{atyyppinen-reisiluun-murtuma}

Valitse yksi:

\begin{itemize}
\tightlist
\item
  \begin{enumerate}
  \def\labelenumi{\alph{enumi}.}
  \tightlist
  \item
    Tavataan nuorilla urheilijoilla rasitusmurtuman komplikaationa
  \end{enumerate}
\item
  \begin{enumerate}
  \def\labelenumi{\alph{enumi}.}
  \setcounter{enumi}{1}
  \tightlist
  \item
    Kuvattu bisfosfonaattihoidon ja denosumabihoidon yhteydessä
  \end{enumerate}
\item
  \begin{enumerate}
  \def\labelenumi{\alph{enumi}.}
  \setcounter{enumi}{2}
  \tightlist
  \item
    Myös oireettomaan murtumaan suositellaan profylaktista naulausta
  \end{enumerate}
\item
  \begin{enumerate}
  \def\labelenumi{\alph{enumi}.}
  \setcounter{enumi}{3}
  \tightlist
  \item
    Matalaenergisissä vammoissa syntyvä osteoporoottinen murtuma
  \end{enumerate}
\end{itemize}

\begin{solution}
\leavevmode

Vastaus

\begin{verbatim}
b
\end{verbatim}

\end{solution}

\section{Eturistisiteen repeämän hoito nuorilla ja keski-ikäisillä}\label{eturistisiteen-repeuxe4muxe4n-hoito-nuorilla-ja-keski-ikuxe4isilluxe4}

Valitse yksi:

\begin{itemize}
\tightlist
\item
  \begin{enumerate}
  \def\labelenumi{\alph{enumi}.}
  \tightlist
  \item
    Leikataan mahdollisimman nopeasti, jotta paraneminen pääsee heti alkuun
  \end{enumerate}
\item
  \begin{enumerate}
  \def\labelenumi{\alph{enumi}.}
  \setcounter{enumi}{1}
  \tightlist
  \item
    Konservatiivisesti hoidettuna potilaalla on suurempi riski saada posttraumaattinen artroosi
  \end{enumerate}
\item
  \begin{enumerate}
  \def\labelenumi{\alph{enumi}.}
  \setcounter{enumi}{2}
  \tightlist
  \item
    Leikkaus ei ole tarpeen, vaikka polvessa olisi muitakin ligamentti- ja kierukkavaurioita
  \end{enumerate}
\item
  \begin{enumerate}
  \def\labelenumi{\alph{enumi}.}
  \setcounter{enumi}{3}
  \tightlist
  \item
    Voidaan hoitaa konservatiivisesti ja leikata tarvittaessa myöhemmin 3-6 kk kuluttua
  \end{enumerate}
\end{itemize}

\begin{solution}
\leavevmode

Vastaus

\begin{verbatim}
d
\end{verbatim}

\begin{itemize}
\item
  Hoito on suunniteltava yksilöllisesti. Alkuvaiheessa kaikille ACL-repeämäpotilaille ohjataan polven liike- ja lihasvoimaharjoitukset. Yksittäisen ACL-repeämän leikkauksellinen hoito on perusteltu, jos reisilihasten aktiivisesta kuntoutuksesta huolimatta polvessa vamman jälkeen on toistuvia muljahduksia tai pettämistuntemuksia. Leikkaushoitoa puoltaa myös fyysisesti raskas työ, liikunnallisesti aktiivinen elämäntapa ja urheiluharrastus, jossa tulee paljon kiihdytyksiä ja suunnanmuutoksia (esim. pallo- ja mailapelit sekä laskettelu). Yksittäisen eturistisiderepeämän rekonstruktioleikkauksella ei ole kiire, se voidaan suorittaa viikkojen tai kuukausien kuluttua.
\item
  c: Eturistisiteen vaurioon liittyy noin puolessa tapauksista traumaattinen kierukkarepeämä. Nämä yhdistelmävammat hoidetaan pääsääntöisesti leikkauksellisesti rekonstruoimalla ACL jännesiirteellä ja ompelemalla kierukkarepeämä muutaman viikon sisällä tapaturmasta. Mikäli traumaattiseen kierukkarepeämään liittyy lukko-oiretta, on suositeltavaa suorittaa leikkaus kiireellisenä. Myös polven moninivelsidevammat suositellaan hoidettavaksi leikkauksellisesti.
\end{itemize}

\section{Mikä seuraavista EI ole nilkan instabiilin (leikkaushoitoa vaativan) lateraalimalleolin murtuman merkki?}\label{mikuxe4-seuraavista-ei-ole-nilkan-instabiilin-leikkaushoitoa-vaativan-lateraalimalleolin-murtuman-merkki}

Valitse yksi:

\begin{itemize}
\tightlist
\item
  \begin{enumerate}
  \def\labelenumi{\alph{enumi}.}
  \tightlist
  \item
    Mediaalinen nivelrako on levinnyt
  \end{enumerate}
\item
  \begin{enumerate}
  \def\labelenumi{\alph{enumi}.}
  \setcounter{enumi}{1}
  \tightlist
  \item
    Lateraalimalleolin murtumassa on ≤ 2 mm sivuttaissiirtymä
  \end{enumerate}
\item
  \begin{enumerate}
  \def\labelenumi{\alph{enumi}.}
  \setcounter{enumi}{2}
  \tightlist
  \item
    Nivelhaarukka on inkongruentti
  \end{enumerate}
\item
  \begin{enumerate}
  \def\labelenumi{\alph{enumi}.}
  \setcounter{enumi}{3}
  \tightlist
  \item
    Kliinisessä kokeessa nilkassa on tunnettavissa sivuttaisliikettä
  \end{enumerate}
\end{itemize}

\begin{solution}
\leavevmode

Vastaus

\begin{verbatim}
b
\end{verbatim}

\begin{itemize}
\item
  Muut ovat instabiliteetin merkkejä -\textgreater{} instabiilit nilkkamurtumat leikataan
\end{itemize}

\section{Olkapään apprehension testissä}\label{olkapuxe4uxe4n-apprehension-testissuxe4}

\pandocbounded{\includegraphics[keepaspectratio]{images/K23_2020.png}}

Valitse yksi:

\begin{itemize}
\tightlist
\item
  \begin{enumerate}
  \def\labelenumi{\alph{enumi}.}
  \tightlist
  \item
    Napsahdus on positiivinen löydös
  \end{enumerate}
\item
  \begin{enumerate}
  \def\labelenumi{\alph{enumi}.}
  \setcounter{enumi}{1}
  \tightlist
  \item
    Kaikki muut vastausvaihtoehdot ovat väärin
  \end{enumerate}
\item
  \begin{enumerate}
  \def\labelenumi{\alph{enumi}.}
  \setcounter{enumi}{2}
  \tightlist
  \item
    Testataan instabiliteettia
  \end{enumerate}
\item
  \begin{enumerate}
  \def\labelenumi{\alph{enumi}.}
  \setcounter{enumi}{3}
  \tightlist
  \item
    Humerus luksoituu posteriorisesti
  \end{enumerate}
\end{itemize}

\begin{solution}
\leavevmode

Vastaus

\begin{verbatim}
c
\end{verbatim}

\end{solution}

\section{Dislokoitumaton patellan murtuma}\label{dislokoitumaton-patellan-murtuma}

Valitse yksi:

\begin{itemize}
\tightlist
\item
  \begin{enumerate}
  \def\labelenumi{\alph{enumi}.}
  \tightlist
  \item
    Voidaan hoitaa 6 viikon immobilisaatiolla (kipsihylsy) sallien varaamisen raajalle 4 viikon kuluttua
  \end{enumerate}
\item
  \begin{enumerate}
  \def\labelenumi{\alph{enumi}.}
  \setcounter{enumi}{1}
  \tightlist
  \item
    Konservatiiviseen hoitoon kuuluu tarvittaessa kivuliaan veripolven punktio ennen kipsausta
  \end{enumerate}
\item
  \begin{enumerate}
  \def\labelenumi{\alph{enumi}.}
  \setcounter{enumi}{2}
  \tightlist
  \item
    Konservatiivisen hoidon tulos on kohtalainen
  \end{enumerate}
\item
  \begin{enumerate}
  \def\labelenumi{\alph{enumi}.}
  \setcounter{enumi}{3}
  \tightlist
  \item
    Kannattaa leikata profylaktisesti dislokaatioriskin takia
  \end{enumerate}
\end{itemize}

\begin{solution}
\leavevmode

Vastaus

\begin{verbatim}
b
\end{verbatim}

\begin{itemize}
\item
  a: Patellamurtuman konservatiivinen hoito on ortoosi/kipsihylsi 3-4 viikkoa ja potilas saa alusta asti kävellessä varata raajaan, kun polvi on ojennettuna. Kons hoito mahdollista, jos polven ojennusvoima on säilynyt eikä lumpion nivelpinnassa ole pykälää tai diastaasia (esim. fissuraalinen poikkimurtuma).
\item
  c: Konservatiivinen hoito usein pettää (jos kyseessä poikittainen murtuma), joten c on väärä vastaus.
\item
  d: Lumpiomurtumia ei suorastaan profylaktisesti leikata (hyväasentoiset poikkimurtumat kuitenkin usein pettävät, joten rtg-kontrollointi on tärkeää).
\end{itemize}

\section{Tämä kulma on}\label{tuxe4muxe4-kulma-on}

\pandocbounded{\includegraphics[keepaspectratio]{images/K25_2020.png}}

Valitse yksi:

\begin{itemize}
\tightlist
\item
  \begin{enumerate}
  \def\labelenumi{\alph{enumi}.}
  \tightlist
  \item
    Volaarinen kallistuskulma
  \end{enumerate}
\item
  \begin{enumerate}
  \def\labelenumi{\alph{enumi}.}
  \setcounter{enumi}{1}
  \tightlist
  \item
    Radiuksen inklinaatio
  \end{enumerate}
\item
  \begin{enumerate}
  \def\labelenumi{\alph{enumi}.}
  \setcounter{enumi}{2}
  \tightlist
  \item
    Dorsaalinen kallistuskulma
  \end{enumerate}
\item
  \begin{enumerate}
  \def\labelenumi{\alph{enumi}.}
  \setcounter{enumi}{3}
  \tightlist
  \item
    Radiuksen pituus
  \end{enumerate}
\end{itemize}

\begin{solution}
\leavevmode

Vastaus

\begin{verbatim}
b
\end{verbatim}

\end{solution}

\section{Solisluun keskikolmanneksen murtuman konservatiivinen hoito}\label{solisluun-keskikolmanneksen-murtuman-konservatiivinen-hoito}

Valitse yksi:
a. 3 viikon kohdalla sallitaan elevaatio vaakatasoon ja 6 viikon jälkeen vapaa mobilisaatio kuormittamatta
b. Kaikki vastausvaihtoehdot ovat oikein
c.~Kantoside 3 viikkoa
d.~Viikon kohdalla aloitetaan pendell-liikeharjoitukset

\begin{solution}
\leavevmode

Vastaus

\begin{verbatim}
b
\end{verbatim}

\end{solution}

\section{Nilkkamurtuma}\label{nilkkamurtuma}

Valitse yksi:

\begin{itemize}
\tightlist
\item
  \begin{enumerate}
  \def\labelenumi{\alph{enumi}.}
  \tightlist
  \item
    Weber C-tyypin murtumissa on yli 50\%:sti syndesmoosin repeämä
  \end{enumerate}
\item
  \begin{enumerate}
  \def\labelenumi{\alph{enumi}.}
  \setcounter{enumi}{1}
  \tightlist
  \item
    Jos lateraalimalleolin murtuman dislokaatio on alle 2 mm, ei leikkaushoito ole tarpeen, vaikka nivelhaarukka näyttäisi hieman leveältä
  \end{enumerate}
\item
  \begin{enumerate}
  \def\labelenumi{\alph{enumi}.}
  \setcounter{enumi}{2}
  \tightlist
  \item
    Jos mediaalimalleolin murtuma ulottuu nivelen sisälle, on leikkaushoito välttämätön
  \end{enumerate}
\item
  \begin{enumerate}
  \def\labelenumi{\alph{enumi}.}
  \setcounter{enumi}{3}
  \tightlist
  \item
    Lateraalimalleolin Weber A-tyypin murtumat voidaan hoitaa yleensä konservatiivisesti
  \end{enumerate}
\end{itemize}

\begin{solution}
\leavevmode

Vastaus

\begin{verbatim}
Taitaa olla jotenkin wikissä väärä kysymyksenasettelu, sillä enemmänkin tässä on vain yksi väärä vastaus (b). Jos olisi pakko valita kuitenkin yksi oikea vastaus, niin se olisi d, koska se on yksiselitteisimmin oikea.
\end{verbatim}

\begin{itemize}
\item
  Weber C -murtumissa on lähes aina syndesmoosin vaurio (käytännössä siis yli 50\%:sti on oikein, mutta hieman harhaanjohtava numero, jonka takia wikissä ajateltu vääräksi vastaukseksi)
\item
  Jos on mediaalimalleolin murtuma, niin leikkaushoito on ensisijainen hoitokeino (ehkä ei ole täysin välttämätön, jonka takia wikissä ajateltu vääräksi vastaukseksi)
\item
  B on väärin, koska leikkaushoito on tarpeen, jos nivelhaarukka ei ole kongruentti (instabiili murtuma)
\end{itemize}

\section{Hauislihaksen jänteen irtoaminen distaalisesta insertiosta}\label{hauislihaksen-juxe4nteen-irtoaminen-distaalisesta-insertiosta}

Valitse yksi:

\begin{itemize}
\tightlist
\item
  \begin{enumerate}
  \def\labelenumi{\alph{enumi}.}
  \tightlist
  \item
    Leikkaukseen liittyy vähän komplikaatioita
  \end{enumerate}
\item
  \begin{enumerate}
  \def\labelenumi{\alph{enumi}.}
  \setcounter{enumi}{1}
  \tightlist
  \item
    Yleisin urheilijoilla, esimerkiksi painonnostajilla
  \end{enumerate}
\item
  \begin{enumerate}
  \def\labelenumi{\alph{enumi}.}
  \setcounter{enumi}{2}
  \tightlist
  \item
    Yleisin ruumiillista työtä tekevillä yli 50 vuotiailla miehillä
  \end{enumerate}
\item
  \begin{enumerate}
  \def\labelenumi{\alph{enumi}.}
  \setcounter{enumi}{3}
  \tightlist
  \item
    Kliininen diagnostiikka vaikeaa, usein sattumalöydös MRI-kuvauksessa
  \end{enumerate}
\end{itemize}

\begin{solution}
\leavevmode

Vastaus

\begin{verbatim}
c
\end{verbatim}

\end{solution}

\section{Fractura radii typica: hyväksyttävä asento reposition jälkeen}\label{fractura-radii-typica-hyvuxe4ksyttuxe4vuxe4-asento-reposition-juxe4lkeen}

\pandocbounded{\includegraphics[keepaspectratio]{images/K29_2020.png}}

Valitse yksi:

\begin{itemize}
\tightlist
\item
  \begin{enumerate}
  \def\labelenumi{\alph{enumi}.}
  \tightlist
  \item
    Lyhentyminen harvoin aiheuttaa toiminnallista haittaa työikäisillä
  \end{enumerate}
\item
  \begin{enumerate}
  \def\labelenumi{\alph{enumi}.}
  \setcounter{enumi}{1}
  \tightlist
  \item
    Kipsi-immobilisaatio estää lyhentymisen syntymisen yli 65-vuotiailla osteoporoosipotilailla
  \end{enumerate}
\item
  \begin{enumerate}
  \def\labelenumi{\alph{enumi}.}
  \setcounter{enumi}{2}
  \tightlist
  \item
    Lyhentyminen vaikuttaa distaalisen radioulnaarinivelen toimintaan
  \end{enumerate}
\item
  \begin{enumerate}
  \def\labelenumi{\alph{enumi}.}
  \setcounter{enumi}{3}
  \tightlist
  \item
    Lyhentymistä ei voida riittävän tarkasti mitata röntgenkuvista
  \end{enumerate}
\end{itemize}

\begin{solution}
\leavevmode

Vastaus

\begin{verbatim}
c
\end{verbatim}

\end{solution}

\section{Olkaluun yläosan murtuma}\label{olkaluun-yluxe4osan-murtuma}

Valitse yksi:

\begin{itemize}
\tightlist
\item
  \begin{enumerate}
  \def\labelenumi{\alph{enumi}.}
  \tightlist
  \item
    Stabiili lukkoruuvilevykiinnitys estää kaputnekroosin kehittymisen dislokoituneissa pirstaleisissa murtumissa
  \end{enumerate}
\item
  \begin{enumerate}
  \def\labelenumi{\alph{enumi}.}
  \setcounter{enumi}{1}
  \tightlist
  \item
    Luksaatiomurtuma on usein ensiavussa helposti reponoitavissa
  \end{enumerate}
\item
  \begin{enumerate}
  \def\labelenumi{\alph{enumi}.}
  \setcounter{enumi}{2}
  \tightlist
  \item
    Pirstaleisten murtumien hoidossa käytettävä käänteinen tekonivel ei vaadi toimivaa kiertäjäkalvosinta
  \end{enumerate}
\item
  \begin{enumerate}
  \def\labelenumi{\alph{enumi}.}
  \setcounter{enumi}{3}
  \tightlist
  \item
    Murtumien ilmaantuvuus on laskussa
  \end{enumerate}
\end{itemize}

\begin{solution}
\leavevmode

Vastaus

\begin{verbatim}
c; Käänteistekonivel ei salli suurta rasitusta, mutta se ei myöskään vaadi toimivaa kiertäjäkalvosinta. Erityisesti vanhusten luksaatiomurtumissa käänteistekonivel suosittu (puolitekonivel nuoremmilla) ja osteosynteesi taas ensisijainen nuorien dislokoituneissa murtumissa. 
\end{verbatim}

\begin{itemize}
\item
  a: Lukitusruuvilevy ei estä kaputnekroosia; se parantaa stabiilisuutta, mutta verenkiertohäiriö voi silti johtaa nekroosiin
\item
  b: luksaatiomurtumat ovat vaikeita reponoida
\item
  d: Olkamurtumien ilmaantuvuus ei ole laskussa (nousussa väestön ikääntyessä)
\end{itemize}

\section{Patellamurtuma}\label{patellamurtuma}

\pandocbounded{\includegraphics[keepaspectratio]{images/K31_2020.png}}

Valitse yksi:

\begin{itemize}
\tightlist
\item
  \begin{enumerate}
  \def\labelenumi{\alph{enumi}.}
  \tightlist
  \item
    Hoidon kannalta nivelpinnan pykälällä ei ole merkitystä
  \end{enumerate}
\item
  \begin{enumerate}
  \def\labelenumi{\alph{enumi}.}
  \setcounter{enumi}{1}
  \tightlist
  \item
    Dislokoitunut murtuma vaatii leikkaushoidon
  \end{enumerate}
\item
  \begin{enumerate}
  \def\labelenumi{\alph{enumi}.}
  \setcounter{enumi}{2}
  \tightlist
  \item
    Tulee pyrkiä reponoimaan ensiavussa
  \end{enumerate}
\item
  \begin{enumerate}
  \def\labelenumi{\alph{enumi}.}
  \setcounter{enumi}{3}
  \tightlist
  \item
    Liitännäisvammana on usein patellajänteen repeämä
  \end{enumerate}
\end{itemize}

\begin{solution}
\leavevmode

Vastaus

\begin{verbatim}
b 
\end{verbatim}

\begin{itemize}
\item
  a: Nivelpinnan pykälällä on merkitystä -\textgreater{} leikkaushoito on yleensä välttämätön
\item
  c: dislokoituneen polvilumpion reponointi ei suorastaan vaikuta hoitovalintaan; alun perin dislokoitunut murtuma leikataan
\item
  d: patellajänteen repeämä liitännäisvammana ei ole kovinkaan yleinen
\end{itemize}

\section{Mikä murtuma?}\label{mikuxe4-murtuma-1}

\pandocbounded{\includegraphics[keepaspectratio]{images/K32_2020.png}}

Valitse yksi:

\begin{itemize}
\tightlist
\item
  \begin{enumerate}
  \def\labelenumi{\alph{enumi}.}
  \tightlist
  \item
    Die punch
  \end{enumerate}
\item
  \begin{enumerate}
  \def\labelenumi{\alph{enumi}.}
  \setcounter{enumi}{1}
  \tightlist
  \item
    Volaarinen Barton
  \end{enumerate}
\item
  \begin{enumerate}
  \def\labelenumi{\alph{enumi}.}
  \setcounter{enumi}{2}
  \tightlist
  \item
    Smith
  \end{enumerate}
\item
  \begin{enumerate}
  \def\labelenumi{\alph{enumi}.}
  \setcounter{enumi}{3}
  \tightlist
  \item
    Dorsaalinen Barton
  \end{enumerate}
\end{itemize}

\begin{solution}
\leavevmode

Vastaus

\begin{verbatim}
d
\end{verbatim}

\end{solution}

\section{Tibialis posterior lihas}\label{tibialis-posterior-lihas}

Valitse yksi:

\begin{itemize}
\tightlist
\item
  \begin{enumerate}
  \def\labelenumi{\alph{enumi}.}
  \tightlist
  \item
    Hoitaa polven koukistusta
  \end{enumerate}
\item
  \begin{enumerate}
  \def\labelenumi{\alph{enumi}.}
  \setcounter{enumi}{1}
  \tightlist
  \item
    Pitää yllä jalkaholvia
  \end{enumerate}
\item
  \begin{enumerate}
  \def\labelenumi{\alph{enumi}.}
  \setcounter{enumi}{2}
  \tightlist
  \item
    Lihaksen jänne repeää usein nilkkamurtuman yhteydessä
  \end{enumerate}
\item
  \begin{enumerate}
  \def\labelenumi{\alph{enumi}.}
  \setcounter{enumi}{3}
  \tightlist
  \item
    Hoitaa polven ojennusta
  \end{enumerate}
\end{itemize}

\begin{solution}
\leavevmode

Vastaus

\begin{verbatim}
b; päätehtäviä on tukea jalan pitkittäistä kaarta (jalkaholvia) sekä osallistua plantaarifleksioon ja invertointiin
\end{verbatim}

\begin{itemize}
\item
  Wikissä väärin (c; ei tyypillisesti repeä nilkkamurtuman yhteydessä)
\item
  Ei koukista tai suorista polvea
\end{itemize}

\section{Fractura radii typica: hyväksyttävä asento reposition jälkeen}\label{fractura-radii-typica-hyvuxe4ksyttuxe4vuxe4-asento-reposition-juxe4lkeen-1}

\pandocbounded{\includegraphics[keepaspectratio]{images/K34_2020.png}}

Valitse yksi:

\begin{itemize}
\tightlist
\item
  \begin{enumerate}
  \def\labelenumi{\alph{enumi}.}
  \tightlist
  \item
    Radiuksen nivelpinta on normaalisti ≥ 20 asteen palmaarisesti kallistuneena
  \end{enumerate}
\item
  \begin{enumerate}
  \def\labelenumi{\alph{enumi}.}
  \setcounter{enumi}{1}
  \tightlist
  \item
    Murtuman reposition jälkeen dorsaalinen kallistuminen pitäisi olla ≤ 5 astetta
  \end{enumerate}
\item
  \begin{enumerate}
  \def\labelenumi{\alph{enumi}.}
  \setcounter{enumi}{2}
  \tightlist
  \item
    Murtuman reposition jälkeen dorsaalinen kallistuminen pitäisi olla ≤ 15 astetta
  \end{enumerate}
\item
  \begin{enumerate}
  \def\labelenumi{\alph{enumi}.}
  \setcounter{enumi}{3}
  \tightlist
  \item
    Radiuksen nivelpinta on normaalisti kohtisuorassa pituusakselia vastaan
  \end{enumerate}
\end{itemize}

\begin{solution}
\leavevmode

Vastaus

\begin{verbatim}
c
\end{verbatim}

\end{solution}

\section{Mikä seuraavista rakenteista on quadricepsjänne?}\label{mikuxe4-seuraavista-rakenteista-on-quadricepsjuxe4nne}

\pandocbounded{\includegraphics[keepaspectratio]{images/K35_2020.png}}

Valitse yksi:

\begin{itemize}
\tightlist
\item
  \begin{enumerate}
  \def\labelenumi{\alph{enumi}.}
  \tightlist
  \item
    3
  \end{enumerate}
\item
  \begin{enumerate}
  \def\labelenumi{\alph{enumi}.}
  \setcounter{enumi}{1}
  \tightlist
  \item
    1
  \end{enumerate}
\item
  \begin{enumerate}
  \def\labelenumi{\alph{enumi}.}
  \setcounter{enumi}{2}
  \tightlist
  \item
    2
  \end{enumerate}
\item
  \begin{enumerate}
  \def\labelenumi{\alph{enumi}.}
  \setcounter{enumi}{3}
  \tightlist
  \item
    4
  \end{enumerate}
\end{itemize}

\begin{solution}
\leavevmode

Vastaus

\begin{verbatim}
b
\end{verbatim}

\end{solution}

\section{Akromioklavikulaarinivelen (AC) luksaatio}\label{akromioklavikulaarinivelen-ac-luksaatio}

Valitse yksi:

\begin{itemize}
\tightlist
\item
  \begin{enumerate}
  \def\labelenumi{\alph{enumi}.}
  \tightlist
  \item
    Paras ja varmasti onnistunut lopputulos saadaan leikkaamalla
  \end{enumerate}
\item
  \begin{enumerate}
  \def\labelenumi{\alph{enumi}.}
  \setcounter{enumi}{1}
  \tightlist
  \item
    Voidaan useimmiten hoitaa konservatiivisesti, vaikka solisluu olisi luksoitunut posteriorisesti (tyyppi IV)
  \end{enumerate}
\item
  \begin{enumerate}
  \def\labelenumi{\alph{enumi}.}
  \setcounter{enumi}{2}
  \tightlist
  \item
    Jos solisluun pää nousee ylöspäin, pitää se leikata 2 viikon sisällä
  \end{enumerate}
\item
  \begin{enumerate}
  \def\labelenumi{\alph{enumi}.}
  \setcounter{enumi}{3}
  \tightlist
  \item
    Tapahtuu usein kaatuessa tai törmätessä olkapää edellä esteeseen
  \end{enumerate}
\end{itemize}

\begin{solution}
\leavevmode

Vastaus

\begin{verbatim}
Sinänsä c ja d ovat molemmat oikein, mutta d on eniten oikein. 
\end{verbatim}

\begin{itemize}
\item
  a: Leikkaushoito ei aina tuota varmasti onnistunutta lopputulosta ja sitä ei myöskään käytetä lievempiasteisissa luksaatioissa (I-III). AC-nivelen täydellinen pysyvä sijoiltaanmeno (tyypit IV - VI) on yleensä tilanne, jossa leikkaushoidon ajatellaan kokemusperäisesti konservatiivista hoitoa hyödyllisemmäksi, vaikka kirjallisuudessa ei olekaan tästä näyttöä.
\item
  b: Useimmiten hoidetaan operatiivisesti, jos tyypin IV luksaatio
\item
  c: Jos solisluun pää on noussut yli luun paksuuden verran tai solisluussa on etu-takasuuntainen epävakaus, voidaan leikkaushoitoa harkita (luokat IV‒V)
\item
  d: AC-luksaatio voi syntyä joko suoran tai epäsuoran vamman seurauksena. Tavallisemmassa eli suorassa vammassa isku tulee yläraajan ollessa kiinni vartalossa hartian ylätakaosaan. Epäsuorassa vammassa isku tulee ojennetun, vartalossa kiinni olevan yläraajan kautta olkaniveleen ja olkaluun pää iskeytyy olkalisäkkeeseen sitä ylöspäin siirtäen. AC-luksaation tavallisin yksittäinen vammamekanismi on polkupyörällä kaatuminen, ja seuraavana tulevat urheiluun liittyvät vammat
\end{itemize}

\section{Haglundin deformiteetti}\label{haglundin-deformiteetti}

\pandocbounded{\includegraphics[keepaspectratio]{images/K37_2020.png}}

Valitse yksi:

\begin{itemize}
\tightlist
\item
  \begin{enumerate}
  \def\labelenumi{\alph{enumi}.}
  \tightlist
  \item
    Kalkaneuksen luinen prominenssi
  \end{enumerate}
\item
  \begin{enumerate}
  \def\labelenumi{\alph{enumi}.}
  \setcounter{enumi}{1}
  \tightlist
  \item
    Taluksen takaosan luinen prominenssi
  \end{enumerate}
\item
  \begin{enumerate}
  \def\labelenumi{\alph{enumi}.}
  \setcounter{enumi}{2}
  \tightlist
  \item
    Yleisin syy akillesjänteen tendinoosiin
  \end{enumerate}
\item
  \begin{enumerate}
  \def\labelenumi{\alph{enumi}.}
  \setcounter{enumi}{3}
  \tightlist
  \item
    Ei näy hyvin röntgenkuvassa, paljastuu tavallisesti MRI-kuvauksessa
  \end{enumerate}
\end{itemize}

\begin{solution}
\leavevmode

Vastaus

\begin{verbatim}
a. Voi esim. aiheuttaa kantapään kipua, varsinkin kun käyttää tiukkoja kenkiä. 
\end{verbatim}

\begin{itemize}
\item
  b: Talus ei liity tähän deformiteettiin
\item
  c: Se ei ole suoraan yleisin syy akillesjänteen tendinoosiin (jänteen rappeumamuutos, joka usein johtuu toistuvasta rasituksesta), vaikka se voi siihen myötävaikuttaa
\item
  d: Näkyy hyvin röntgenkuvassa, koska on luinen prominenssi
\end{itemize}

\section{Smithin murtuma}\label{smithin-murtuma}

\pandocbounded{\includegraphics[keepaspectratio]{images/K38_2020.png}}

Valitse yksi:

\begin{itemize}
\tightlist
\item
  \begin{enumerate}
  \def\labelenumi{\alph{enumi}.}
  \tightlist
  \item
    Reposition jälkeen kipsaus pehmustetulla kulmakipsilastalla ranne lievässä volaarifleksiossa ja pronaatiossa
  \end{enumerate}
\item
  \begin{enumerate}
  \def\labelenumi{\alph{enumi}.}
  \setcounter{enumi}{1}
  \tightlist
  \item
    Distaalinen radiusmurtuma, jossa distaalifragmentti on dorsaalisesti kallistuneena
  \end{enumerate}
\item
  \begin{enumerate}
  \def\labelenumi{\alph{enumi}.}
  \setcounter{enumi}{2}
  \tightlist
  \item
    Soveltuu hyvin konservatiiviseen hoitoon
  \end{enumerate}
\item
  \begin{enumerate}
  \def\labelenumi{\alph{enumi}.}
  \setcounter{enumi}{3}
  \tightlist
  \item
    Leikkaushoito tarpeen useimmilla potilailla
  \end{enumerate}
\end{itemize}

\begin{solution}
\leavevmode

Vastaus

\begin{verbatim}
d
\end{verbatim}

\begin{itemize}
\item
  a: Rannemurtumat lastoitetaan funktioasentoon (lievä ekstensio ja muuten neutraaliasento). Smithin murtumassa lastoitus on volaarinen (vrt. Collesin murtuma, jossa dorsaalinen lasta)
\item
  b: Tämä olisi COllesin murtuma (tyyppimurtuma). Smithin murtumassa fragmentti on volaarisesti kallistunut
\item
  c: Soveltuu konservatiiviseen hoitoon tietyissä tapauksissa, mutta usein vaatii leikkaushoidon.
\end{itemize}

\section{Tässä ranteessa on}\label{tuxe4ssuxe4-ranteessa-on}

\pandocbounded{\includegraphics[keepaspectratio]{images/K39_2020.png}}

Valitse yksi:

\begin{itemize}
\tightlist
\item
  \begin{enumerate}
  \def\labelenumi{\alph{enumi}.}
  \tightlist
  \item
    Bartonin murtuma
  \end{enumerate}
\item
  \begin{enumerate}
  \def\labelenumi{\alph{enumi}.}
  \setcounter{enumi}{1}
  \tightlist
  \item
    Tyypillinen radiusmurtuma (Colles)
  \end{enumerate}
\item
  \begin{enumerate}
  \def\labelenumi{\alph{enumi}.}
  \setcounter{enumi}{2}
  \tightlist
  \item
    Chaffeurin murtuma
  \end{enumerate}
\item
  \begin{enumerate}
  \def\labelenumi{\alph{enumi}.}
  \setcounter{enumi}{3}
  \tightlist
  \item
    Smithin murtuma
  \end{enumerate}
\end{itemize}

\begin{solution}
\leavevmode

Vastaus

\begin{verbatim}
b
\end{verbatim}

\end{solution}

\section{Olkaluun yläosan murtuma}\label{olkaluun-yluxe4osan-murtuma-1}

Valitse yksi:

\begin{itemize}
\tightlist
\item
  \begin{enumerate}
  \def\labelenumi{\alph{enumi}.}
  \tightlist
  \item
    Konservatiivinen hoito johtaa usein huonoon toiminnalliseen lopputulokseen
  \end{enumerate}
\item
  \begin{enumerate}
  \def\labelenumi{\alph{enumi}.}
  \setcounter{enumi}{1}
  \tightlist
  \item
    Yli 5 mm dislokaatio isoloidussa tuberculum majuksen murtumassa vaatii leikkaushoitoa
  \end{enumerate}
\item
  \begin{enumerate}
  \def\labelenumi{\alph{enumi}.}
  \setcounter{enumi}{2}
  \tightlist
  \item
    Johtaa usein nopeasti (noin 1-2 kuukaudessa) rtg-kuvassa näkyvään kaputnekroosiin
  \end{enumerate}
\item
  \begin{enumerate}
  \def\labelenumi{\alph{enumi}.}
  \setcounter{enumi}{3}
  \tightlist
  \item
    Yli 60\% olkaluun yläosan murtumista on huonoasentoisia vaatien leikkaushoitoa
  \end{enumerate}
\end{itemize}

\begin{solution}
\leavevmode

Vastaus

\begin{verbatim}
b
\end{verbatim}

\begin{itemize}
\item
  Dislokoitumattoman tai lievästi dislokoituneen (\textless{} 5 mm) murtuman hoito on konservatiivinen. Raaja pidetään enintään 3 viikkoa mitellassa. ps. Muiden olkapään Neerin fragmenttien dislokoitumisrajana tyypillisesti pidetään 10mm
\item
  a: Konservatiivinen hoito johtaa useimmiten hyvään lopputulokseen niissä murtumissa, joissa se on indikoitu
\item
  c: Kaputnekroosi on harvinainen ja liittyy pääasiassa monilohkomurtumiin (esim. 3--4 osainen murtuma) ja murtumaluksaatioihin, ei yksittäiseen tuberculum majuksen murtumaan
\item
  d: lähteestä riippuen n.~70-80\% olkaluun yläosan murtumista on hyväasentoisia murtumia
\end{itemize}

\section{Avomurtumat}\label{avomurtumat}

Valitse yksi:

\begin{itemize}
\tightlist
\item
  \begin{enumerate}
  \def\labelenumi{\alph{enumi}.}
  \tightlist
  \item
    Gradus IIIB: kirurgista korjausta vaativa verisuonivamma
  \end{enumerate}
\item
  \begin{enumerate}
  \def\labelenumi{\alph{enumi}.}
  \setcounter{enumi}{1}
  \tightlist
  \item
    Gradus IIIC: traumaattinen amputaatio
  \end{enumerate}
\item
  c.Gradus IIIA: Laaja iho- ja lihaskudosvamma, joka vaatii lihaskielekkeen
\item
  \begin{enumerate}
  \def\labelenumi{\alph{enumi}.}
  \setcounter{enumi}{3}
  \tightlist
  \item
    Gradus I vamma: terävä luufragmentti perforoi ihon sisältä ulospäin
  \end{enumerate}
\end{itemize}

\begin{solution}
\leavevmode

Vastaus

\begin{verbatim}
d
\end{verbatim}

\begin{itemize}
\item
  a: Avomurtuman yhteydessä korjausta vaativa verisuonivamma = Gustilo-Andersonin luokka IIIc (IIIb = vaatii kielekerekonstruktion)
\item
  b: Gradus IIIC: korjausta vaativa verisuonivamma (Gustilo-Andersonin luokitus ei kuvaile amputaatiovammoja)
\item
  c: Gradus IIIa: laaja pehmytkudosvaurio, mutta se ei vaadi kielekerekonstruktiota
\end{itemize}

\section{Polven ojentajamekanismiin kuuluvat}\label{polven-ojentajamekanismiin-kuuluvat}

Valitse yksi:

\begin{itemize}
\tightlist
\item
  \begin{enumerate}
  \def\labelenumi{\alph{enumi}.}
  \tightlist
  \item
    Tuberositas tibiae, patellajänne, patella, quadricepsjänne ja -lihas
  \end{enumerate}
\item
  \begin{enumerate}
  \def\labelenumi{\alph{enumi}.}
  \setcounter{enumi}{1}
  \tightlist
  \item
    Posterolateraalinurkka ja takakapseli
  \end{enumerate}
\item
  c.ACL, PCL ja mediaalimeniski
\item
  \begin{enumerate}
  \def\labelenumi{\alph{enumi}.}
  \setcounter{enumi}{3}
  \tightlist
  \item
    Quadricepsjänne, LCL ja patella
  \end{enumerate}
\end{itemize}

\begin{solution}
\leavevmode

Vastaus

\begin{verbatim}
a
\end{verbatim}

\end{solution}

\section{Katkennut akillesjänne}\label{katkennut-akillesjuxe4nne}

Valitse yksi:

\begin{itemize}
\tightlist
\item
  \begin{enumerate}
  \def\labelenumi{\alph{enumi}.}
  \tightlist
  \item
    Voidaan yleensä hoitaa konservatiivisesti
  \end{enumerate}
\item
  \begin{enumerate}
  \def\labelenumi{\alph{enumi}.}
  \setcounter{enumi}{1}
  \tightlist
  \item
    Kaikki vastausvaihtoehdot ovat oikein
  \end{enumerate}
\item
  \begin{enumerate}
  \def\labelenumi{\alph{enumi}.}
  \setcounter{enumi}{2}
  \tightlist
  \item
    Joka 3. potilaalla esioireita ja diagnoosi viivästyy joka 4. potilaalla
  \end{enumerate}
\item
  \begin{enumerate}
  \def\labelenumi{\alph{enumi}.}
  \setcounter{enumi}{3}
  \tightlist
  \item
    Yleensä poikki 3-4 cm kantaluuinsertion yläpuolelta
  \end{enumerate}
\end{itemize}

\begin{solution}
\leavevmode

Vastaus

\begin{verbatim}
b
\end{verbatim}

\begin{itemize}
\item
  a: Akillesjänteen repeämien hoidossa suositaan konservatiivista hoitoa, koska ortoosilla toteutetulla hoidolla saavutetaan käytännössä sama paranemistulos kuin operatiivisella hoidolla
\item
  b: Potilailla löytyy akillesjänteestä lähes aina degeneratiivisia muutoksia, mutta 2/3:ssa tapauksista katkeaminen tapahtuu ilman minkäänlaisia edeltäviä oireita. Diagnostiikassa petollista on se, että usein vähäinen nilkan ekstensio-fleksioliike jää kuormittamattomana jäljelle -\textgreater{} diagnoosi viivästyy.
\item
  d: Tyypillisesti poikkeaa ns. akilleksen watershed-alueelta, jossa verenkierto on heikointa. Tämä sattuu olemaan n.~2-6cm insertiokohdan yläpuolella (lähteestä riippuen senttimäärät hieman vaihtelevat)
\end{itemize}

\section{Olkaluun yläosan murtuman konservatiivinen hoito}\label{olkaluun-yluxe4osan-murtuman-konservatiivinen-hoito}

Valitse yksi:

\begin{itemize}
\tightlist
\item
  \begin{enumerate}
  \def\labelenumi{\alph{enumi}.}
  \tightlist
  \item
    Yläraaja tuetaan 2-3 viikoksi kantositeeseen ja kevyet heiluriliikkeet aloitetaan viikon sisällä
  \end{enumerate}
\item
  \begin{enumerate}
  \def\labelenumi{\alph{enumi}.}
  \setcounter{enumi}{1}
  \tightlist
  \item
    Aktiiviset kiertoliike ja abduktioharjoitukset aloitetaan 3-5 viikon jälkeen
  \end{enumerate}
\item
  \begin{enumerate}
  \def\labelenumi{\alph{enumi}.}
  \setcounter{enumi}{2}
  \tightlist
  \item
    Kaikki vastausvaihtoehdot ovat oikein
  \end{enumerate}
\item
  \begin{enumerate}
  \def\labelenumi{\alph{enumi}.}
  \setcounter{enumi}{3}
  \tightlist
  \item
    Tavallisin komplikaatio on olkanivelen kipu ja jäykistyminen
  \end{enumerate}
\end{itemize}

\begin{solution}
\leavevmode

Vastaus

\begin{verbatim}
c
\end{verbatim}

\end{solution}

\section{Kuinka kuvan vamma tulisi ensisijaisesti hoitaa?}\label{kuinka-kuvan-vamma-tulisi-ensisijaisesti-hoitaa}

\pandocbounded{\includegraphics[keepaspectratio]{images/K45_2020.png}}

Valitse yksi:

\begin{itemize}
\tightlist
\item
  \begin{enumerate}
  \def\labelenumi{\alph{enumi}.}
  \tightlist
  \item
    Repositio ja kantoside
  \end{enumerate}
\item
  \begin{enumerate}
  \def\labelenumi{\alph{enumi}.}
  \setcounter{enumi}{1}
  \tightlist
  \item
    Repositio ja kulmakipsi
  \end{enumerate}
\item
  \begin{enumerate}
  \def\labelenumi{\alph{enumi}.}
  \setcounter{enumi}{2}
  \tightlist
  \item
    Repositio ja vapaa mobilisaatio
  \end{enumerate}
\item
  \begin{enumerate}
  \def\labelenumi{\alph{enumi}.}
  \setcounter{enumi}{3}
  \tightlist
  \item
    Leikkaus
  \end{enumerate}
\end{itemize}

\begin{solution}
\leavevmode

Vastaus

\begin{verbatim}
c
\end{verbatim}

\begin{itemize}
\item
  Kyynärnivelluksaation primaarihoito tutkimuksen jälkeen on suljettu paikalleenasettaminen. Koska suurimpaan osaan kyynärnivelen sijoiltaanmenoista ei liity merkittäviä nivelsidevammoja, on tämä yleensä riittävä hoito
\item
  Kyynärnivelen tulisi pysyä paikallaan ojennus-koukistussuunnassa sektorilla 60‒130, jotta konservatiivisella hoidolla on edellytykset onnistua. Operatiivisen hoidon indikaatioita ovat repositionjälkeinen instabiliteetti fleksio-ekstensiosuunnassa, intra-artikulaarinen murtumafragmentti, avoluksaatio tai uusintaluksaatio konservatiivisessa hoidossa.
\end{itemize}

\section{Syndesmoosin tehtävä on}\label{syndesmoosin-tehtuxe4vuxe4-on}

Valitse yksi:

\begin{itemize}
\tightlist
\item
  \begin{enumerate}
  \def\labelenumi{\alph{enumi}.}
  \tightlist
  \item
    Tarjota verenkierto fibulan distaaliosaan ja tibian mediaaliosiin
  \end{enumerate}
\item
  \begin{enumerate}
  \def\labelenumi{\alph{enumi}.}
  \setcounter{enumi}{1}
  \tightlist
  \item
    Kiinnittää talus TC-niveleen
  \end{enumerate}
\item
  \begin{enumerate}
  \def\labelenumi{\alph{enumi}.}
  \setcounter{enumi}{2}
  \tightlist
  \item
    Stabiloida säären yläosa ja mahdollistaa polven optimaalinen liikerata
  \end{enumerate}
\item
  \begin{enumerate}
  \def\labelenumi{\alph{enumi}.}
  \setcounter{enumi}{3}
  \tightlist
  \item
    Pitää tibia ja fibula toiminnallisesti yhdessä, jotta nilkkahaarukka pysyy kongruenttina fysiologisen kuormituksen aikana
  \end{enumerate}
\end{itemize}

\begin{solution}
\leavevmode

Vastaus

\begin{verbatim}
d
\end{verbatim}

\end{solution}

\section{Quadricepsjänteen repeämä}\label{quadricepsjuxe4nteen-repeuxe4muxe4}

Valitse yksi:

\begin{itemize}
\tightlist
\item
  \begin{enumerate}
  \def\labelenumi{\alph{enumi}.}
  \tightlist
  \item
    Aiheuttaa liikeheikkouden polven koukistukseen potilaan ollessa vatsamakuulla
  \end{enumerate}
\item
  \begin{enumerate}
  \def\labelenumi{\alph{enumi}.}
  \setcounter{enumi}{1}
  \tightlist
  \item
    Johtuu yleensä suorasta iskusta polven seutuun
  \end{enumerate}
\item
  \begin{enumerate}
  \def\labelenumi{\alph{enumi}.}
  \setcounter{enumi}{2}
  \tightlist
  \item
    Voidaan yleensä hoitaa konservatiivisesti
  \end{enumerate}
\item
  \begin{enumerate}
  \def\labelenumi{\alph{enumi}.}
  \setcounter{enumi}{3}
  \tightlist
  \item
    Tuntuu usein palpoituvana kuoppana polvilumpion proksimaalipuolella ja potilas ei pysty nostamaan raajaa ojennettuna
  \end{enumerate}
\end{itemize}

\begin{solution}
\leavevmode

Vastaus

\begin{verbatim}
d
\end{verbatim}

\end{solution}

\section{Mikä seuraavista rakenteista on processus coracoideus?}\label{mikuxe4-seuraavista-rakenteista-on-processus-coracoideus}

\pandocbounded{\includegraphics[keepaspectratio]{images/K48_2020.png}}

Valitse yksi:

\begin{itemize}
\tightlist
\item
  \begin{enumerate}
  \def\labelenumi{\alph{enumi}.}
  \tightlist
  \item
    1
  \end{enumerate}
\item
  \begin{enumerate}
  \def\labelenumi{\alph{enumi}.}
  \setcounter{enumi}{1}
  \tightlist
  \item
    2
  \end{enumerate}
\item
  \begin{enumerate}
  \def\labelenumi{\alph{enumi}.}
  \setcounter{enumi}{2}
  \tightlist
  \item
    4
  \end{enumerate}
\item
  \begin{enumerate}
  \def\labelenumi{\alph{enumi}.}
  \setcounter{enumi}{3}
  \tightlist
  \item
    3
  \end{enumerate}
\end{itemize}

\begin{solution}
\leavevmode

Vastaus

\begin{verbatim}
b
\end{verbatim}

\end{solution}

\section{Hamstring jännerepeämä}\label{hamstring-juxe4nnerepeuxe4muxe4}

Valitse yksi:

\begin{itemize}
\tightlist
\item
  \begin{enumerate}
  \def\labelenumi{\alph{enumi}.}
  \tightlist
  \item
    Testataan vastustetulla lonkan fleksiolla
  \end{enumerate}
\item
  \begin{enumerate}
  \def\labelenumi{\alph{enumi}.}
  \setcounter{enumi}{1}
  \tightlist
  \item
    Kaikki muut vastausvaihtoehdot ovat väärin
  \end{enumerate}
\item
  \begin{enumerate}
  \def\labelenumi{\alph{enumi}.}
  \setcounter{enumi}{2}
  \tightlist
  \item
    Aiheutuu vammautuneen alaraajan voimakkaasta lonkan ekstensiosta
  \end{enumerate}
\item
  \begin{enumerate}
  \def\labelenumi{\alph{enumi}.}
  \setcounter{enumi}{3}
  \tightlist
  \item
    Totaaliruptuura hoidetaan useimmiten konservatiivisesti
  \end{enumerate}
\end{itemize}

\begin{solution}
\leavevmode

Vastaus

\begin{verbatim}
b
\end{verbatim}

\begin{itemize}
\item
  a: Hamstringit eivät suorita lonkan fleksiota, vaan lonkan ekstensiota ja polven koukistusta.
\item
  c: Aiheutuu vammautuneen alaraajan voimakkaasta lonkan fleksiosta (esim. liukastuminen spagaattiin)
\item
  d: Totaaliruptuura hoidetaan useimmiten leikkauksellisesti. Irronnut lihasryhmä istutetaan takaisin irtoamakohtaansa (istuinkyhmy) yleensä ankkurikiinnitteisin langoin. Hoito olisi hyvä toteuttaa ensimmäisten viikkojen aikana (kuukauden sisällä), sillä lihaksen lyhentymä voi estää myöhäisvaiheessa jänteen palauttamisen paikalleen ja vaurion aiheuttama kiinnikkeisyys ympäristöön, erityisesti iskiashermoon tai sen haaroihin, tekee myöhäisvaiheen leikkauksesta merkittävästi haastavampaa ja riskialttiimpaa
\end{itemize}

\chapter{2021 (Invictus)}\label{invictus}

Tähän yhteyteen laitettu vain ne kysymykset, joita ei ollut aiemmassa vuoden 2020 tentissä.

\section{Miten tämän potilaan malletsormi (vasarasormi) hoidetaan?}\label{miten-tuxe4muxe4n-potilaan-malletsormi-vasarasormi-hoidetaan}

\pandocbounded{\includegraphics[keepaspectratio]{images/K1_2021.png}}

Valitse yksi:

\begin{itemize}
\tightlist
\item
  \begin{enumerate}
  \def\labelenumi{\alph{enumi}.}
  \tightlist
  \item
    Lastoitus 2vk ajan
  \end{enumerate}
\item
  \begin{enumerate}
  \def\labelenumi{\alph{enumi}.}
  \setcounter{enumi}{1}
  \tightlist
  \item
    Lastoitus 6vk ajan
  \end{enumerate}
\item
  \begin{enumerate}
  \def\labelenumi{\alph{enumi}.}
  \setcounter{enumi}{2}
  \tightlist
  \item
    Teippaus viereiseen sormeen 4vk ajan
  \end{enumerate}
\item
  \begin{enumerate}
  \def\labelenumi{\alph{enumi}.}
  \setcounter{enumi}{3}
  \tightlist
  \item
    Leikkaushoito
  \end{enumerate}
\end{itemize}

\begin{solution}
\leavevmode

Vastaus

\begin{verbatim}
b
\end{verbatim}

\begin{itemize}
\item
  Ojentajajänteen vaurio sormen DIP-nivelseudussa aiheuttaa kärkijäsenen ojennusvajauksen (mallet-deformiteetti). Vamma voi syntyä suljettuna tai avoimena joko jänteen katkeamisena tai liittyä distaalifalangin murtumaan. Tässä yhteydessä nyt ei ole murtumaa todettavissa.
\item
  Hoidetaan tyypillisesti konservatiivisesti DIP-nivel lastalla hyperekstensioon n.~6 viikon ajaksi. Hoidon aikana DIP-niveltä ei saa koukistaa. Lastoitusta kannattaa kokeilla, vaikka hoidon aloitus olisi viivästynyt. Jatkohoitona lastaa kannattaa pitää vielä 8 viikon ajan öisin.
\item
  Leikkaushoito on indikoitua, jos mallet-fracturessa fragmentti ei reponoidu, sen koko käsittää lähteestä riippuen 1/3 tai 1/2 tai enemmän nivelpinnasta ja diastaasi on yli 2 mm
\end{itemize}

\section{Oheisen kuvan potilaalla on}\label{oheisen-kuvan-potilaalla-on}

\pandocbounded{\includegraphics[keepaspectratio]{images/K3_2021.png}}

Valitse yksi:

\begin{itemize}
\tightlist
\item
  \begin{enumerate}
  \def\labelenumi{\alph{enumi}.}
  \tightlist
  \item
    Lunotriquetraalisen ligamentin repeämä
  \end{enumerate}
\item
  \begin{enumerate}
  \def\labelenumi{\alph{enumi}.}
  \setcounter{enumi}{1}
  \tightlist
  \item
    Skafolunaarisen ligamentin repeämä
  \end{enumerate}
\item
  \begin{enumerate}
  \def\labelenumi{\alph{enumi}.}
  \setcounter{enumi}{2}
  \tightlist
  \item
    Perilunaarinen luksaatio
  \end{enumerate}
\item
  \begin{enumerate}
  \def\labelenumi{\alph{enumi}.}
  \setcounter{enumi}{3}
  \tightlist
  \item
    Veneluun murtuma
  \end{enumerate}
\end{itemize}

\begin{solution}
\leavevmode

Vastaus

\begin{verbatim}
b
\end{verbatim}

\begin{itemize}
\item
  Skafolunaariligamentin vaurio on yleisin ranteen ligamenttivaurio. Vamma syntyy yleisimmin kaaduttaessa ja potilaan ottaessa vastaan ojennetulla kädellä. Ligamenttivamma voi olla itsenäinen tai liittyä esimerkiksi radius- tai skafoideummurtumaan.
\item
  Röntgenissä näkyy SL-välin leveäminen (tyypillisesti alle 2mm)
\item
  Epäiltäessä skafolunaarista vauriota potilaan lähettämisessä käsikirurgiseen arvioon ei ole syytä viivytellä, sillä nivelsidekorjaus voidaan yleensä tehdä akuutissa vaiheessa 4‒6 viikon sisällä vammasta. Hoitamattomana SL-välin avautuminen voi jatkua ja johtaa etenevään artroosiin (SLAC = scaphoid lunate advanced collapse), jonka hoitona on ranteen osittainen tai täydellinen luudutus.
\end{itemize}

\section{Osteosarkooma}\label{osteosarkooma}

Valitse yksi:

\begin{itemize}
\tightlist
\item
  \begin{enumerate}
  \def\labelenumi{\alph{enumi}.}
  \tightlist
  \item
    On tavallisin maligni rustotuumori
  \end{enumerate}
\item
  \begin{enumerate}
  \def\labelenumi{\alph{enumi}.}
  \setcounter{enumi}{1}
  \tightlist
  \item
    Ennuste 30\% yli 5 vuotta
  \end{enumerate}
\item
  \begin{enumerate}
  \def\labelenumi{\alph{enumi}.}
  \setcounter{enumi}{2}
  \tightlist
  \item
    Hoitona pre- ja postoperatiivinen systostaatti, radikaaliresektio ja rekonstruktio
  \end{enumerate}
\item
  \begin{enumerate}
  \def\labelenumi{\alph{enumi}.}
  \setcounter{enumi}{3}
  \tightlist
  \item
    Yleisin yli 65-vuotiailla, metafyysialue
  \end{enumerate}
\end{itemize}

\begin{solution}
\leavevmode

Vastaus

\begin{verbatim}
c
\end{verbatim}

\begin{itemize}
\item
  a: On tavallisin primaarinen maligni luutuumori. Tavallisin primaarinen maligni rustoa tuottava tuumori on kondrosarkooma.
\item
  b: 5-vuotiselossaoloennuste on n.~70-80\% (parantunut huomattavasti viime vuosikymmeninä). Ennuste tosin on alle 30\% metastaattisessa taudissa
\item
  d: Yleisin lapsilla ja nuorilla aikuisilla (useimmiten alle 25v). Ilmenee vastauksen mukaan kuitenkin useimmiten metafyysialueilla (varsinkin polven ympärillä eli distaalisessa femurissa tai proksimaalisessa tibiassa).
\end{itemize}

\section{Nilkkamurtuma}\label{nilkkamurtuma-1}

Valitse yksi:

\begin{itemize}
\tightlist
\item
  \begin{enumerate}
  \def\labelenumi{\alph{enumi}.}
  \tightlist
  \item
    Weber C-tyypin murtumissa syndesmoosi ei käytännössä repeä
  \end{enumerate}
\item
  \begin{enumerate}
  \def\labelenumi{\alph{enumi}.}
  \setcounter{enumi}{1}
  \tightlist
  \item
    Lateraalimalleolin Weber A-tyypin murtumat voidaan hoitaa yleensä konservatiivisesti
  \end{enumerate}
\item
  \begin{enumerate}
  \def\labelenumi{\alph{enumi}.}
  \setcounter{enumi}{2}
  \tightlist
  \item
    Mediaalimalleolin murtuma liittyy aina Weber B-tyypin murtumaan
  \end{enumerate}
\item
  \begin{enumerate}
  \def\labelenumi{\alph{enumi}.}
  \setcounter{enumi}{3}
  \tightlist
  \item
    Jos lateraalimalleolin murtuman dislokaatio on alle 2mm, ei leikkaushoito ole tarpeen, vaikka nivelhaarukka näyttäisi hieman leveältä
  \end{enumerate}
\end{itemize}

\begin{solution}
\leavevmode

Vastaus

\begin{verbatim}
b
\end{verbatim}

\begin{itemize}
\item
  a: Weber C-murtumissa syndesmoosi käytännössä aina repeää.\\
\item
  b: Weber A-murtumat ovat yleensä stabiileja, jonka takia ne voidaan hoitaa konservatiivisesti (tyypillisesti ortoosi tai kipsihoito 1-3 viikkoa; varaus kivun mukaan)
\item
  c: Mediaalimalleolin murtuma voi esiintyä monissa yhdistelmissä, mutta useimmiten tapahtuvat Weber C-murtumien yhteydessä.
\item
  d: Leikkaushoito on tarpeen, jos nivel on instabiili ja yksi tämän merkki on inkongruentti nivelhaarukka röntgenissä.
\end{itemize}

\section{Olkaluun yläosan murtuma}\label{olkaluun-yluxe4osan-murtuma-2}

Valitse yksi:

\begin{itemize}
\tightlist
\item
  \begin{enumerate}
  \def\labelenumi{\alph{enumi}.}
  \tightlist
  \item
    Stabiili lukkoruuvilevytys estää kaputnekroosin kehittymisen dislokoituneissa pirstaleisissa murtumissa
  \end{enumerate}
\item
  \begin{enumerate}
  \def\labelenumi{\alph{enumi}.}
  \setcounter{enumi}{1}
  \tightlist
  \item
    Murtumien ilmaantuvuus on laskussa
  \end{enumerate}
\item
  \begin{enumerate}
  \def\labelenumi{\alph{enumi}.}
  \setcounter{enumi}{2}
  \tightlist
  \item
    Pirstaleisten murtumien hoidossa käänteinen tekonivel on yleistynyt viime vuosina
  \end{enumerate}
\item
  \begin{enumerate}
  \def\labelenumi{\alph{enumi}.}
  \setcounter{enumi}{3}
  \tightlist
  \item
    Luksaatiomurtuma on usein ensiavussa helposti reponoitavissa
  \end{enumerate}
\end{itemize}

\begin{solution}
\leavevmode

Vastaus

\begin{verbatim}
c; Käänteistekonivel ei salli suurta rasitusta, mutta se ei myöskään vaadi toimivaa kiertäjäkalvosinta. Erityisesti vanhusten luksaatiomurtumissa käänteistekonivel suosittu (puolitekonivel nuoremmilla) ja osteosynteesi taas ensisijainen nuorien dislokoituneissa murtumissa. 
\end{verbatim}

\begin{itemize}
\item
  a: Lukitusruuvilevy ei estä kaputnekroosia; se parantaa stabiilisuutta, mutta verenkiertohäiriö voi silti johtaa nekroosiin
\item
  b: Olkamurtumien ilmaantuvuus ei ole laskussa (nousussa väestön ikääntyessä)
\item
  d: Luksaatiomurtumat ovat vaikeita reponoida
\end{itemize}

\section{Viidennen kämmenluun kaulan (Boxerin) murtuman yleisin hoitomuoto:}\label{viidennen-kuxe4mmenluun-kaulan-boxerin-murtuman-yleisin-hoitomuoto}

Valitse yksi

\begin{itemize}
\tightlist
\item
  \begin{enumerate}
  \def\labelenumi{\alph{enumi}.}
  \tightlist
  \item
    Seuranta ilman toimenpiteitä
  \end{enumerate}
\item
  \begin{enumerate}
  \def\labelenumi{\alph{enumi}.}
  \setcounter{enumi}{1}
  \tightlist
  \item
    Leikkaushoito
  \end{enumerate}
\item
  \begin{enumerate}
  \def\labelenumi{\alph{enumi}.}
  \setcounter{enumi}{2}
  \tightlist
  \item
    Pikkurilli ja nimetön kipsataan SAFE-asentoon
  \end{enumerate}
\item
  \begin{enumerate}
  \def\labelenumi{\alph{enumi}.}
  \setcounter{enumi}{3}
  \tightlist
  \item
    Pikkurilli teipataan viereiseen sormeen
  \end{enumerate}
\end{itemize}

\begin{solution}
\leavevmode

Vastaus

\begin{verbatim}
d
\end{verbatim}

\begin{itemize}
\item
  Jos Boxerin murtumassa todetaan hyväksyttävä asento (nivel on kongruentti ja angulaatio alle 45-60 astetta eikä kiertovirhettä) ja potilas saa ojennettua sormen suoraksi, niin murtuma hoidetaan konservatiivisesti teippaamalla sormi viereiseen nimettömään n.~4 viikon ajaksi
\end{itemize}

\section{Solisluun murtumat}\label{solisluun-murtumat-1}

Valitse yksi

\begin{itemize}
\tightlist
\item
  \begin{enumerate}
  \def\labelenumi{\alph{enumi}.}
  \tightlist
  \item
    Keksikolmanneksen murtumat ovat yleisimpiä
  \end{enumerate}
\item
  \begin{enumerate}
  \def\labelenumi{\alph{enumi}.}
  \setcounter{enumi}{1}
  \tightlist
  \item
    Keskikolmanneksen murtuamt hoidetaan yleensä konservatiivisesti
  \end{enumerate}
\item
  \begin{enumerate}
  \def\labelenumi{\alph{enumi}.}
  \setcounter{enumi}{2}
  \tightlist
  \item
    Kaikki kysymyksen vastausvaihtoehdot ovat oikein
  \end{enumerate}
\item
  \begin{enumerate}
  \def\labelenumi{\alph{enumi}.}
  \setcounter{enumi}{3}
  \tightlist
  \item
    Leikkausta harkitaan, jos dislokaatiota murtumassa on yli luunpaksuuden verran
  \end{enumerate}
\end{itemize}

\begin{solution}
\leavevmode

Vastaus

\begin{verbatim}
c
\end{verbatim}

\end{solution}

\section{Hiihtäjän peukalo}\label{hiihtuxe4juxe4n-peukalo}

Valitse yksi

\begin{itemize}
\tightlist
\item
  \begin{enumerate}
  \def\labelenumi{\alph{enumi}.}
  \tightlist
  \item
    Hoidetaan yleensä ilman röntgenkuvausta
  \end{enumerate}
\item
  \begin{enumerate}
  \def\labelenumi{\alph{enumi}.}
  \setcounter{enumi}{1}
  \tightlist
  \item
    On peukalon koukistajanjänteen sulkeinen repeämä
  \end{enumerate}
\item
  \begin{enumerate}
  \def\labelenumi{\alph{enumi}.}
  \setcounter{enumi}{2}
  \tightlist
  \item
    Vaatii leikkaushoitoa, jos nivel on epätukeva
  \end{enumerate}
\item
  \begin{enumerate}
  \def\labelenumi{\alph{enumi}.}
  \setcounter{enumi}{3}
  \tightlist
  \item
    Aiheutuu peukalon metakarpofalangeaali (MCP)-nivelen vääntymisestä ulnaarideviaatioon
  \end{enumerate}
\end{itemize}

\begin{solution}
\leavevmode

Vastaus

\begin{verbatim}
c
\end{verbatim}

\begin{itemize}
\item
  a: Vamma tulee kuvantaa
\item
  b: Hiihtäjänpeukalo = Peukalon MCP-nivelen ulnaarisen kollateraaliligamentin osittainen tai täydellinen vamma. Syntyy abdusoivan voiman vaikutuksesta eli tyypillisesti peukalon vääntyessä radiaalisesti esim. kaaduttaessa hiihtäessä.
\item
  c: Nivelen stabiliteetti on tutkittava tarkasti. Kivun takia voi olla alkuvaiheessa vaikeaa saada luotettavaa käsitystä nivelsiteen stabiliteetista. Tarvittaessa sormi puudutetaan luotettavan testaustuloksen saamiseksi tai voi immobilisoida peukalon kipsilastalla tai ortoosilla ja tutkia nivelside uudelleen muutaman päivän kuluttua. Täydelliseen repeämään viittaavat \textgreater{} 30°:n periksianto radiaalisuuntaan, \textgreater{} 15°:n ero terveeseen puoleen verrattuna ja liikealan napakan päätepisteen (end-point) puuttuminen. Instabiili nivelside hoidetaan operatiivisesti. Leikkauksen jälkeen peukalo immobilisoidaan kyynärvarteen ulottuvalla dorsaalisella kipsilastalla tai ortoosilla 6 viikon ajan.
\item
  d: Aiheutuu peukalon metakarpofalangeaali (MCP)-nivelen vääntymisestä radiaalideviaatioon. Peukalon MCP-nivelen radiaalikollateraalivamma, joka on edellistä huomattavasti harvinaisempi, syntyy taas ulnaarideviaatiossa. Hoidetaan samojen periaatteiden mukaisesti.
\end{itemize}

\section{Luunmurtuman paraneminen}\label{luunmurtuman-paraneminen-1}

Valitse yksi

\begin{itemize}
\tightlist
\item
  \begin{enumerate}
  \def\labelenumi{\alph{enumi}.}
  \tightlist
  \item
    Fractura itineraria syntyy tyypillisesti laskettelurinteessä
  \end{enumerate}
\item
  \begin{enumerate}
  \def\labelenumi{\alph{enumi}.}
  \setcounter{enumi}{1}
  \tightlist
  \item
    Yläraajan kierteinen murtuma luutuu yleensä 12 viikossa
  \end{enumerate}
\item
  \begin{enumerate}
  \def\labelenumi{\alph{enumi}.}
  \setcounter{enumi}{2}
  \tightlist
  \item
    Putkiluu paranee nopeammin kuin hohkaluu
  \end{enumerate}
\item
  \begin{enumerate}
  \def\labelenumi{\alph{enumi}.}
  \setcounter{enumi}{3}
  \tightlist
  \item
    Consolidatio tarda tarkoittaa hidastunutta luutumista
  \end{enumerate}
\end{itemize}

\begin{solution}
\leavevmode

Vastaus

\begin{verbatim}
d
\end{verbatim}

\begin{itemize}
\item
  a: Fractura itineraria = rasitusmurtuma (fatigue fracture); murtumia, joille on tyypillistä kumuloiva syklinen kuormitus, joka johtaa asteittain täydelliseen murtumaviivaan muuten suhteellisen mekaanisesti terveessä luussa
\item
  b: Yläraajan kierteinen murtuma luutuu yleensä 6 viikossa (lujittuu 3 viikossa); Yläraajan poikkimurtuma taas tyypillisesti luutuu 12 viikossa
\item
  c: Hohkaluu paranee nopeammin. Hohkaluun hyvä verenkierto ja paikallisten mesenkymaalisten kantasolujen runsas määrä takaavat yleensä murtuman nopean paranemisen. Metafyysimurtumat siis luutuvat nopeammin kuin diafyysimurtumat.
\item
  d: Hidastunut luutuminen (delayed union, consolidatio tarda) tarkoittaa, että 3 kk kohdalla ei havaita kliinisiä/radiologisia merkkejä luutumisen käynnistymisestä (todetaan hetkuminen ja kipu, raaja ei kestä kuormitusta)
\end{itemize}

\chapter{2022 (Cupidus)}\label{cupidus}

\section{Reisiluun distaaliset murtumat}\label{reisiluun-distaaliset-murtumat}

\chapter{Blocks}\label{blocks}

\section{Equations}\label{equations}

Here is an equation.

\begin{equation} 
  f\left(k\right) = \binom{n}{k} p^k\left(1-p\right)^{n-k}
  \label{eq:binom}
\end{equation}

You may refer to using \texttt{\textbackslash{}@ref(eq:binom)}, like see Equation \eqref{eq:binom}.

\section{Theorems and proofs}\label{theorems-and-proofs}

Labeled theorems can be referenced in text using \texttt{\textbackslash{}@ref(thm:tri)}, for example, check out this smart theorem \ref{thm:tri}.

\begin{theorem}
\protect\hypertarget{thm:tri}{}\label{thm:tri}For a right triangle, if \(c\) denotes the \emph{length} of the hypotenuse
and \(a\) and \(b\) denote the lengths of the \textbf{other} two sides, we have
\[a^2 + b^2 = c^2\]
\end{theorem}

Read more here \url{https://bookdown.org/yihui/bookdown/markdown-extensions-by-bookdown.html}.

\section{Callout blocks}\label{callout-blocks}

The R Markdown Cookbook provides more help on how to use custom blocks to design your own callouts: \url{https://bookdown.org/yihui/rmarkdown-cookbook/custom-blocks.html}

\chapter{Sharing your book}\label{sharing-your-book}

\section{Publishing}\label{publishing}

HTML books can be published online, see: \url{https://bookdown.org/yihui/bookdown/publishing.html}

\section{404 pages}\label{pages}

By default, users will be directed to a 404 page if they try to access a webpage that cannot be found. If you'd like to customize your 404 page instead of using the default, you may add either a \texttt{\_404.Rmd} or \texttt{\_404.md} file to your project root and use code and/or Markdown syntax.

\section{Metadata for sharing}\label{metadata-for-sharing}

Bookdown HTML books will provide HTML metadata for social sharing on platforms like Twitter, Facebook, and LinkedIn, using information you provide in the \texttt{index.Rmd} YAML. To setup, set the \texttt{url} for your book and the path to your \texttt{cover-image} file. Your book's \texttt{title} and \texttt{description} are also used.

This \texttt{gitbook} uses the same social sharing data across all chapters in your book- all links shared will look the same.

Specify your book's source repository on GitHub using the \texttt{edit} key under the configuration options in the \texttt{\_output.yml} file, which allows users to suggest an edit by linking to a chapter's source file.

Read more about the features of this output format here:

\url{https://pkgs.rstudio.com/bookdown/reference/gitbook.html}

Or use:

\begin{Shaded}
\begin{Highlighting}[]
\NormalTok{?bookdown}\SpecialCharTok{::}\NormalTok{gitbook}
\end{Highlighting}
\end{Shaded}

\end{solution}

\end{solution}

\end{solution}

\end{solution}

\end{solution}

\end{solution}

\end{solution}

\end{solution}

\end{solution}

\end{solution}

\end{solution}

\end{solution}

\end{solution}

\end{solution}

\end{solution}

\end{solution}

\end{solution}

\end{solution}

\end{solution}

\end{solution}

\end{solution}

\end{solution}

\end{solution}

\end{solution}

\end{solution}

\end{solution}

\end{solution}

\end{solution}

\end{solution}

\bibliography{book.bib,packages.bib}

\end{document}
